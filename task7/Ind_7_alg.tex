 \documentclass[11pt]{report}

\usepackage[T2A]{fontenc}

\usepackage[utf8]{inputenc}

\usepackage[russian]{babel}

\usepackage{amsmath,amssymb}

\usepackage{graphicx}

\graphicspath{ {d:/HSE/OR/CW/CW5pict/} }

\begin{document}

\pagestyle{empty}

{\bf Индивидуальное задание.}

Вариант N 1

Даны точки: K (2, 1, -1), M (1, 4, 0), N (-11, -5, 15), P (-3, 1, 5), Q (3, -2, 4).
Составить словарь с ключами - точками (например, Point3D(3, 2, 1)) и значениями - именами точек (K, M и т.д.).

 
Найти и вывести на экран


расстояние от точек Q и N до плоскости KMP,

 
уравнение плоскости KMP,

 
параметрические уравнения прямой QN.


Определить и вывести на экран, какие 4 из заданных точек лежат в одной плоскости (пользуясь словарем точек, вывести имена точек).


Определить и вывести на экран, какие 3 из заданных точек лежат на одной прямой (пользуясь словарем точек, вывести имена точек).

Вариант N 2

Даны точки: G (5, 3, -1), K (-2, -1, -3), Q (20, 0, 5), S (0, 4, -3), T (4, -3, 0).
Составить словарь с ключами - точками (например, Point3D(3, 2, 1)) и значениями - именами точек (G, S и т.д.).

 
Найти и вывести на экран


расстояние от точек K и Q до плоскости GST,

 
уравнение плоскости GST,

 
параметрические уравнения прямой KQ.


Определить и вывести на экран, какие 4 из заданных точек лежат в одной плоскости (пользуясь словарем точек, вывести имена точек).


Определить и вывести на экран, какие 3 из заданных точек лежат на одной прямой (пользуясь словарем точек, вывести имена точек).

Вариант N 3

Даны точки: J (3, -9, -21), L (3, 3, 3), M (3, 1, -1), N (2, -2, 2), Q (-2, 5, -2).
Составить словарь с ключами - точками (например, Point3D(3, 2, 1)) и значениями - именами точек (L, M и т.д.).

 
Найти и вывести на экран


расстояние от точек N и J до плоскости LMQ,

 
уравнение плоскости LMQ,

 
параметрические уравнения прямой NJ.


Определить и вывести на экран, какие 4 из заданных точек лежат в одной плоскости (пользуясь словарем точек, вывести имена точек).


Определить и вывести на экран, какие 3 из заданных точек лежат на одной прямой (пользуясь словарем точек, вывести имена точек).

Вариант N 4

Даны точки: G (0, 0, 2), M (5, 4, 5), P (1, -2, 3), Q (-1, -16, 9), T (2, 5, 0).
Составить словарь с ключами - точками (например, Point3D(3, 2, 1)) и значениями - именами точек (M, P и т.д.).

 
Найти и вывести на экран


расстояние от точек G и Q до плоскости MPT,

 
уравнение плоскости MPT,

 
параметрические уравнения прямой GQ.


Определить и вывести на экран, какие 4 из заданных точек лежат в одной плоскости (пользуясь словарем точек, вывести имена точек).


Определить и вывести на экран, какие 3 из заданных точек лежат на одной прямой (пользуясь словарем точек, вывести имена точек).

Вариант N 5

Даны точки: G (-3, -2, 1), L (-2, 5, -4), M (2, 3, 5), N (-1, 3, 8), P (5, 3, 2).
Составить словарь с ключами - точками (например, Point3D(3, 2, 1)) и значениями - именами точек (L, M и т.д.).

 
Найти и вывести на экран


расстояние от точек G и N до плоскости LMP,

 
уравнение плоскости LMP,

 
параметрические уравнения прямой GN.


Определить и вывести на экран, какие 4 из заданных точек лежат в одной плоскости (пользуясь словарем точек, вывести имена точек).


Определить и вывести на экран, какие 3 из заданных точек лежат на одной прямой (пользуясь словарем точек, вывести имена точек).

Вариант N 6

Даны точки: F (-8, -10, 12), G (-4, -2, 4), H (-2, 2, 0), L (-3, 5, 2), Q (-4, 5, 3).
Составить словарь с ключами - точками (например, Point3D(3, 2, 1)) и значениями - именами точек (G, H и т.д.).

 
Найти и вывести на экран


расстояние от точек Q и F до плоскости GHL,

 
уравнение плоскости GHL,

 
параметрические уравнения прямой QF.


Определить и вывести на экран, какие 4 из заданных точек лежат в одной плоскости (пользуясь словарем точек, вывести имена точек).


Определить и вывести на экран, какие 3 из заданных точек лежат на одной прямой (пользуясь словарем точек, вывести имена точек).

Вариант N 7

Даны точки: F (-3, -2, -2), M (2, 5, 2), P (2, -2, 3), R (2, 12, -5), T (2, 5, -1).
Составить словарь с ключами - точками (например, Point3D(3, 2, 1)) и значениями - именами точек (M, P и т.д.).

 
Найти и вывести на экран


расстояние от точек F и R до плоскости MPT,

 
уравнение плоскости MPT,

 
параметрические уравнения прямой FR.


Определить и вывести на экран, какие 4 из заданных точек лежат в одной плоскости (пользуясь словарем точек, вывести имена точек).


Определить и вывести на экран, какие 3 из заданных точек лежат на одной прямой (пользуясь словарем точек, вывести имена точек).

Вариант N 8

Даны точки: F (-3, 0, -1), K (-3, 4, -1), P (3, 0, 2), Q (0, 1, -4), R (-21, 16, -10).
Составить словарь с ключами - точками (например, Point3D(3, 2, 1)) и значениями - именами точек (F, K и т.д.).

 
Найти и вывести на экран


расстояние от точек Q и R до плоскости FKP,

 
уравнение плоскости FKP,

 
параметрические уравнения прямой QR.


Определить и вывести на экран, какие 4 из заданных точек лежат в одной плоскости (пользуясь словарем точек, вывести имена точек).


Определить и вывести на экран, какие 3 из заданных точек лежат на одной прямой (пользуясь словарем точек, вывести имена точек).

Вариант N 9

Даны точки: H (5, -3, -3), N (20, 0, -12), P (0, -4, 0), S (-2, 5, 4), T (-1, -1, 1).
Составить словарь с ключами - точками (например, Point3D(3, 2, 1)) и значениями - именами точек (H, P и т.д.).

 
Найти и вывести на экран


расстояние от точек S и N до плоскости HPT,

 
уравнение плоскости HPT,

 
параметрические уравнения прямой SN.


Определить и вывести на экран, какие 4 из заданных точек лежат в одной плоскости (пользуясь словарем точек, вывести имена точек).


Определить и вывести на экран, какие 3 из заданных точек лежат на одной прямой (пользуясь словарем точек, вывести имена точек).

Вариант N 10

Даны точки: G (0, -3, 3), J (-9, -18, -3), K (2, -3, 2), P (3, 2, 5), R (0, -1, -3).
Составить словарь с ключами - точками (например, Point3D(3, 2, 1)) и значениями - именами точек (G, K и т.д.).

 
Найти и вывести на экран


расстояние от точек R и J до плоскости GKP,

 
уравнение плоскости GKP,

 
параметрические уравнения прямой RJ.


Определить и вывести на экран, какие 4 из заданных точек лежат в одной плоскости (пользуясь словарем точек, вывести имена точек).


Определить и вывести на экран, какие 3 из заданных точек лежат на одной прямой (пользуясь словарем точек, вывести имена точек).

Вариант N 11

Даны точки: H (0, 5, 0), K (5, 0, -4), M (2, -4, -3), P (0, -2, -3), R (0, 33, 12).
Составить словарь с ключами - точками (например, Point3D(3, 2, 1)) и значениями - именами точек (H, M и т.д.).

 
Найти и вывести на экран


расстояние от точек K и R до плоскости HMP,

 
уравнение плоскости HMP,

 
параметрические уравнения прямой KR.


Определить и вывести на экран, какие 4 из заданных точек лежат в одной плоскости (пользуясь словарем точек, вывести имена точек).


Определить и вывести на экран, какие 3 из заданных точек лежат на одной прямой (пользуясь словарем точек, вывести имена точек).

Вариант N 12

Даны точки: L (4, 3, 4), Q (-4, 4, 1), R (-17, -9, -11), S (-3, -1, -1), T (3, 2, -3).
Составить словарь с ключами - точками (например, Point3D(3, 2, 1)) и значениями - именами точек (L, S и т.д.).

 
Найти и вывести на экран


расстояние от точек Q и R до плоскости LST,

 
уравнение плоскости LST,

 
параметрические уравнения прямой QR.


Определить и вывести на экран, какие 4 из заданных точек лежат в одной плоскости (пользуясь словарем точек, вывести имена точек).


Определить и вывести на экран, какие 3 из заданных точек лежат на одной прямой (пользуясь словарем точек, вывести имена точек).

Вариант N 13

Даны точки: G (-1, -4, 3), H (0, 1, -1), M (3, 3, 3), N (20, -9, 9), T (4, -1, 1).
Составить словарь с ключами - точками (например, Point3D(3, 2, 1)) и значениями - именами точек (H, M и т.д.).

 
Найти и вывести на экран


расстояние от точек G и N до плоскости HMT,

 
уравнение плоскости HMT,

 
параметрические уравнения прямой GN.


Определить и вывести на экран, какие 4 из заданных точек лежат в одной плоскости (пользуясь словарем точек, вывести имена точек).


Определить и вывести на экран, какие 3 из заданных точек лежат на одной прямой (пользуясь словарем точек, вывести имена точек).

Вариант N 14

Даны точки: K (0, -2, 5), L (5, 0, -1), M (-3, -4, 2), N (21, 12, 14), T (3, 0, 5).
Составить словарь с ключами - точками (например, Point3D(3, 2, 1)) и значениями - именами точек (L, M и т.д.).

 
Найти и вывести на экран


расстояние от точек K и N до плоскости LMT,

 
уравнение плоскости LMT,

 
параметрические уравнения прямой KN.


Определить и вывести на экран, какие 4 из заданных точек лежат в одной плоскости (пользуясь словарем точек, вывести имена точек).


Определить и вывести на экран, какие 3 из заданных точек лежат на одной прямой (пользуясь словарем точек, вывести имена точек).

Вариант N 15

Даны точки: F (2, 4, -4), H (4, -2, -4), P (-4, -3, -3), R (44, 3, -9), T (2, 5, 4).
Составить словарь с ключами - точками (например, Point3D(3, 2, 1)) и значениями - именами точек (H, P и т.д.).

 
Найти и вывести на экран


расстояние от точек F и R до плоскости HPT,

 
уравнение плоскости HPT,

 
параметрические уравнения прямой FR.


Определить и вывести на экран, какие 4 из заданных точек лежат в одной плоскости (пользуясь словарем точек, вывести имена точек).


Определить и вывести на экран, какие 3 из заданных точек лежат на одной прямой (пользуясь словарем точек, вывести имена точек).

Вариант N 16

Даны точки: J (7, 18, -10), K (2, -4, -2), L (3, 4, -4), N (-4, -3, 1), P (1, -3, -1).
Составить словарь с ключами - точками (например, Point3D(3, 2, 1)) и значениями - именами точек (K, L и т.д.).

 
Найти и вывести на экран


расстояние от точек N и J до плоскости KLP,

 
уравнение плоскости KLP,

 
параметрические уравнения прямой NJ.


Определить и вывести на экран, какие 4 из заданных точек лежат в одной плоскости (пользуясь словарем точек, вывести имена точек).


Определить и вывести на экран, какие 3 из заданных точек лежат на одной прямой (пользуясь словарем точек, вывести имена точек).

Вариант N 17

Даны точки: K (4, -3, -2), M (5, -3, 1), Q (-2, 4, -4), R (17, -35, 13), S (2, 5, -2).
Составить словарь с ключами - точками (например, Point3D(3, 2, 1)) и значениями - именами точек (K, M и т.д.).

 
Найти и вывести на экран


расстояние от точек Q и R до плоскости KMS,

 
уравнение плоскости KMS,

 
параметрические уравнения прямой QR.


Определить и вывести на экран, какие 4 из заданных точек лежат в одной плоскости (пользуясь словарем точек, вывести имена точек).


Определить и вывести на экран, какие 3 из заданных точек лежат на одной прямой (пользуясь словарем точек, вывести имена точек).

Вариант N 18

Даны точки: L (4, 1, -1), N (6, 7, -1), R (4, -3, 5), S (3, -2, -1), T (-2, 5, -4).
Составить словарь с ключами - точками (например, Point3D(3, 2, 1)) и значениями - именами точек (L, S и т.д.).

 
Найти и вывести на экран


расстояние от точек R и N до плоскости LST,

 
уравнение плоскости LST,

 
параметрические уравнения прямой RN.


Определить и вывести на экран, какие 4 из заданных точек лежат в одной плоскости (пользуясь словарем точек, вывести имена точек).


Определить и вывести на экран, какие 3 из заданных точек лежат на одной прямой (пользуясь словарем точек, вывести имена точек).

Вариант N 19

Даны точки: G (7, 7, -2), L (2, 5, 5), M (4, 4, 0), S (1, 1, 2), T (0, -3, -4).
Составить словарь с ключами - точками (например, Point3D(3, 2, 1)) и значениями - именами точек (L, M и т.д.).

 
Найти и вывести на экран


расстояние от точек T и G до плоскости LMS,

 
уравнение плоскости LMS,

 
параметрические уравнения прямой TG.


Определить и вывести на экран, какие 4 из заданных точек лежат в одной плоскости (пользуясь словарем точек, вывести имена точек).


Определить и вывести на экран, какие 3 из заданных точек лежат на одной прямой (пользуясь словарем точек, вывести имена точек).

Вариант N 20

Даны точки: G (-3, 1, -4), L (3, 4, 5), N (13, 13, 28), Q (4, 2, 0), S (1, 4, 4).
Составить словарь с ключами - точками (например, Point3D(3, 2, 1)) и значениями - именами точек (G, L и т.д.).

 
Найти и вывести на экран


расстояние от точек Q и N до плоскости GLS,

 
уравнение плоскости GLS,

 
параметрические уравнения прямой QN.


Определить и вывести на экран, какие 4 из заданных точек лежат в одной плоскости (пользуясь словарем точек, вывести имена точек).


Определить и вывести на экран, какие 3 из заданных точек лежат на одной прямой (пользуясь словарем точек, вывести имена точек).

Вариант N 21

Даны точки: F (-3, -3, -3), G (-2, -1, -1), K (-1, 3, 2), N (-7, -21, -16), Q (1, 0, -2).
Составить словарь с ключами - точками (например, Point3D(3, 2, 1)) и значениями - именами точек (F, G и т.д.).

 
Найти и вывести на экран


расстояние от точек Q и N до плоскости FGK,

 
уравнение плоскости FGK,

 
параметрические уравнения прямой QN.


Определить и вывести на экран, какие 4 из заданных точек лежат в одной плоскости (пользуясь словарем точек, вывести имена точек).


Определить и вывести на экран, какие 3 из заданных точек лежат на одной прямой (пользуясь словарем точек, вывести имена точек).

Вариант N 22

Даны точки: G (1, -1, -4), K (-3, -2, 5), L (5, 5, -2), M (-3, -1, 0), R (17, -1, -20).
Составить словарь с ключами - точками (например, Point3D(3, 2, 1)) и значениями - именами точек (G, K и т.д.).

 
Найти и вывести на экран


расстояние от точек L и R до плоскости GKM,

 
уравнение плоскости GKM,

 
параметрические уравнения прямой LR.


Определить и вывести на экран, какие 4 из заданных точек лежат в одной плоскости (пользуясь словарем точек, вывести имена точек).


Определить и вывести на экран, какие 3 из заданных точек лежат на одной прямой (пользуясь словарем точек, вывести имена точек).

Вариант N 23

Даны точки: F (-2, 3, 5), G (-2, 0, 5), P (-1, -3, 4), Q (3, 11, 6), S (1, 4, 5).
Составить словарь с ключами - точками (например, Point3D(3, 2, 1)) и значениями - именами точек (G, P и т.д.).

 
Найти и вывести на экран


расстояние от точек F и Q до плоскости GPS,

 
уравнение плоскости GPS,

 
параметрические уравнения прямой FQ.


Определить и вывести на экран, какие 4 из заданных точек лежат в одной плоскости (пользуясь словарем точек, вывести имена точек).


Определить и вывести на экран, какие 3 из заданных точек лежат на одной прямой (пользуясь словарем точек, вывести имена точек).

Вариант N 24

Даны точки: F (-3, 5, -1), G (-4, 2, 3), J (-22, -7, 3), M (2, 5, 3), R (-4, -1, -1).
Составить словарь с ключами - точками (например, Point3D(3, 2, 1)) и значениями - именами точек (F, G и т.д.).

 
Найти и вывести на экран


расстояние от точек R и J до плоскости FGM,

 
уравнение плоскости FGM,

 
параметрические уравнения прямой RJ.


Определить и вывести на экран, какие 4 из заданных точек лежат в одной плоскости (пользуясь словарем точек, вывести имена точек).


Определить и вывести на экран, какие 3 из заданных точек лежат на одной прямой (пользуясь словарем точек, вывести имена точек).

Вариант N 25

Даны точки: H (3, 3, 4), K (4, 0, 4), N (5, 1, 2), P (2, 4, 5), S (3, 2, -1).
Составить словарь с ключами - точками (например, Point3D(3, 2, 1)) и значениями - именами точек (H, P и т.д.).

 
Найти и вывести на экран


расстояние от точек K и N до плоскости HPS,

 
уравнение плоскости HPS,

 
параметрические уравнения прямой KN.


Определить и вывести на экран, какие 4 из заданных точек лежат в одной плоскости (пользуясь словарем точек, вывести имена точек).


Определить и вывести на экран, какие 3 из заданных точек лежат на одной прямой (пользуясь словарем точек, вывести имена точек).

Вариант N 26

Даны точки: F (1, -2, 5), H (4, 5, 3), M (1, -3, 1), N (19, 45, 13), R (5, 2, 0).
Составить словарь с ключами - точками (например, Point3D(3, 2, 1)) и значениями - именами точек (F, H и т.д.).

 
Найти и вывести на экран


расстояние от точек R и N до плоскости FHM,

 
уравнение плоскости FHM,

 
параметрические уравнения прямой RN.


Определить и вывести на экран, какие 4 из заданных точек лежат в одной плоскости (пользуясь словарем точек, вывести имена точек).


Определить и вывести на экран, какие 3 из заданных точек лежат на одной прямой (пользуясь словарем точек, вывести имена точек).

Вариант N 27

Даны точки: F (-1, -2, 3), J (10, 32, -2), K (0, 0, 0), L (-2, -4, -2), M (0, 2, -2).
Составить словарь с ключами - точками (например, Point3D(3, 2, 1)) и значениями - именами точек (K, L и т.д.).

 
Найти и вывести на экран


расстояние от точек F и J до плоскости KLM,

 
уравнение плоскости KLM,

 
параметрические уравнения прямой FJ.


Определить и вывести на экран, какие 4 из заданных точек лежат в одной плоскости (пользуясь словарем точек, вывести имена точек).


Определить и вывести на экран, какие 3 из заданных точек лежат на одной прямой (пользуясь словарем точек, вывести имена точек).

Вариант N 28

Даны точки: K (4, 0, 1), L (-2, -3, 0), M (0, -3, 5), R (-10, -8, -20), T (-2, -4, 0).
Составить словарь с ключами - точками (например, Point3D(3, 2, 1)) и значениями - именами точек (L, M и т.д.).

 
Найти и вывести на экран


расстояние от точек K и R до плоскости LMT,

 
уравнение плоскости LMT,

 
параметрические уравнения прямой KR.


Определить и вывести на экран, какие 4 из заданных точек лежат в одной плоскости (пользуясь словарем точек, вывести имена точек).


Определить и вывести на экран, какие 3 из заданных точек лежат на одной прямой (пользуясь словарем точек, вывести имена точек).

Вариант N 29

Даны точки: F (2, -4, 0), H (2, 0, -2), M (-1, -1, 5), N (-2, 0, 2), P (3, 0, -3).
Составить словарь с ключами - точками (например, Point3D(3, 2, 1)) и значениями - именами точек (H, M и т.д.).

 
Найти и вывести на экран


расстояние от точек F и N до плоскости HMP,

 
уравнение плоскости HMP,

 
параметрические уравнения прямой FN.


Определить и вывести на экран, какие 4 из заданных точек лежат в одной плоскости (пользуясь словарем точек, вывести имена точек).


Определить и вывести на экран, какие 3 из заданных точек лежат на одной прямой (пользуясь словарем точек, вывести имена точек).

Вариант N 30

Даны точки: J (5, 10, -3), L (5, 2, -3), M (5, 0, -3), N (2, 4, 2), Q (-1, -2, 0).
Составить словарь с ключами - точками (например, Point3D(3, 2, 1)) и значениями - именами точек (L, M и т.д.).

 
Найти и вывести на экран


расстояние от точек N и J до плоскости LMQ,

 
уравнение плоскости LMQ,

 
параметрические уравнения прямой NJ.


Определить и вывести на экран, какие 4 из заданных точек лежат в одной плоскости (пользуясь словарем точек, вывести имена точек).


Определить и вывести на экран, какие 3 из заданных точек лежат на одной прямой (пользуясь словарем точек, вывести имена точек).

Вариант N 31

Даны точки: K (-2, 0, -2), L (0, 0, 5), P (1, 1, 3), R (-9, 5, -3), T (-4, 3, 0).
Составить словарь с ключами - точками (например, Point3D(3, 2, 1)) и значениями - именами точек (L, P и т.д.).

 
Найти и вывести на экран


расстояние от точек K и R до плоскости LPT,

 
уравнение плоскости LPT,

 
параметрические уравнения прямой KR.


Определить и вывести на экран, какие 4 из заданных точек лежат в одной плоскости (пользуясь словарем точек, вывести имена точек).


Определить и вывести на экран, какие 3 из заданных точек лежат на одной прямой (пользуясь словарем точек, вывести имена точек).

Вариант N 32

Даны точки: J (-33, -14, 13), L (2, 3, 1), M (-4, 5, -2), P (-3, -4, 3), S (3, -2, 1).
Составить словарь с ключами - точками (например, Point3D(3, 2, 1)) и значениями - именами точек (L, P и т.д.).

 
Найти и вывести на экран


расстояние от точек M и J до плоскости LPS,

 
уравнение плоскости LPS,

 
параметрические уравнения прямой MJ.


Определить и вывести на экран, какие 4 из заданных точек лежат в одной плоскости (пользуясь словарем точек, вывести имена точек).


Определить и вывести на экран, какие 3 из заданных точек лежат на одной прямой (пользуясь словарем точек, вывести имена точек).

Вариант N 33

Даны точки: G (-4, -4, 5), H (2, -1, -4), M (-2, 1, 4), N (7, -6, -4), S (3, -2, -4).
Составить словарь с ключами - точками (например, Point3D(3, 2, 1)) и значениями - именами точек (G, H и т.д.).

 
Найти и вывести на экран


расстояние от точек M и N до плоскости GHS,

 
уравнение плоскости GHS,

 
параметрические уравнения прямой MN.


Определить и вывести на экран, какие 4 из заданных точек лежат в одной плоскости (пользуясь словарем точек, вывести имена точек).


Определить и вывести на экран, какие 3 из заданных точек лежат на одной прямой (пользуясь словарем точек, вывести имена точек).

Вариант N 34

Даны точки: F (1, -1, -3), G (1, -4, -2), J (-8, 6, 11), M (-2, 3, 5), S (0, 2, 3).
Составить словарь с ключами - точками (например, Point3D(3, 2, 1)) и значениями - именами точек (G, M и т.д.).

 
Найти и вывести на экран


расстояние от точек F и J до плоскости GMS,

 
уравнение плоскости GMS,

 
параметрические уравнения прямой FJ.


Определить и вывести на экран, какие 4 из заданных точек лежат в одной плоскости (пользуясь словарем точек, вывести имена точек).


Определить и вывести на экран, какие 3 из заданных точек лежат на одной прямой (пользуясь словарем точек, вывести имена точек).

Вариант N 35

Даны точки: F (-3, 5, -1), G (4, -4, 4), K (-4, -1, -3), L (3, 4, 4), N (-1, 36, 4).
Составить словарь с ключами - точками (например, Point3D(3, 2, 1)) и значениями - именами точек (G, K и т.д.).

 
Найти и вывести на экран


расстояние от точек F и N до плоскости GKL,

 
уравнение плоскости GKL,

 
параметрические уравнения прямой FN.


Определить и вывести на экран, какие 4 из заданных точек лежат в одной плоскости (пользуясь словарем точек, вывести имена точек).


Определить и вывести на экран, какие 3 из заданных точек лежат на одной прямой (пользуясь словарем точек, вывести имена точек).

Вариант N 36

Даны точки: G (0, -4, -1), H (4, -4, 4), J (4, 16, -1), M (-3, -2, -3), T (4, 0, 3).
Составить словарь с ключами - точками (например, Point3D(3, 2, 1)) и значениями - именами точек (G, H и т.д.).

 
Найти и вывести на экран


расстояние от точек M и J до плоскости GHT,

 
уравнение плоскости GHT,

 
параметрические уравнения прямой MJ.


Определить и вывести на экран, какие 4 из заданных точек лежат в одной плоскости (пользуясь словарем точек, вывести имена точек).


Определить и вывести на экран, какие 3 из заданных точек лежат на одной прямой (пользуясь словарем точек, вывести имена точек).

Вариант N 37

Даны точки: F (-6, 12, 0), G (-4, 2, 0), H (-3, -3, 0), K (4, -3, -4), M (-2, 1, 2).
Составить словарь с ключами - точками (например, Point3D(3, 2, 1)) и значениями - именами точек (G, H и т.д.).

 
Найти и вывести на экран


расстояние от точек K и F до плоскости GHM,

 
уравнение плоскости GHM,

 
параметрические уравнения прямой KF.


Определить и вывести на экран, какие 4 из заданных точек лежат в одной плоскости (пользуясь словарем точек, вывести имена точек).


Определить и вывести на экран, какие 3 из заданных точек лежат на одной прямой (пользуясь словарем точек, вывести имена точек).

Вариант N 38

Даны точки: G (-2, 2, 4), J (10, 5, -5), K (3, 1, 0), N (0, 3, 5), P (2, 3, 1).
Составить словарь с ключами - точками (например, Point3D(3, 2, 1)) и значениями - именами точек (G, K и т.д.).

 
Найти и вывести на экран


расстояние от точек N и J до плоскости GKP,

 
уравнение плоскости GKP,

 
параметрические уравнения прямой NJ.


Определить и вывести на экран, какие 4 из заданных точек лежат в одной плоскости (пользуясь словарем точек, вывести имена точек).


Определить и вывести на экран, какие 3 из заданных точек лежат на одной прямой (пользуясь словарем точек, вывести имена точек).

Вариант N 39

Даны точки: F (4, 1, 4), J (12, 11, -31), P (-4, 3, 5), Q (-1, 0, 0), S (0, 5, -4).
Составить словарь с ключами - точками (например, Point3D(3, 2, 1)) и значениями - именами точек (P, Q и т.д.).

 
Найти и вывести на экран


расстояние от точек F и J до плоскости PQS,

 
уравнение плоскости PQS,

 
параметрические уравнения прямой FJ.


Определить и вывести на экран, какие 4 из заданных точек лежат в одной плоскости (пользуясь словарем точек, вывести имена точек).


Определить и вывести на экран, какие 3 из заданных точек лежат на одной прямой (пользуясь словарем точек, вывести имена точек).

Вариант N 40

Даны точки: K (2, 0, -2), L (2, 0, 3), M (3, -3, -2), N (24, -12, -14), S (-4, 0, 2).
Составить словарь с ключами - точками (например, Point3D(3, 2, 1)) и значениями - именами точек (L, M и т.д.).

 
Найти и вывести на экран


расстояние от точек K и N до плоскости LMS,

 
уравнение плоскости LMS,

 
параметрические уравнения прямой KN.


Определить и вывести на экран, какие 4 из заданных точек лежат в одной плоскости (пользуясь словарем точек, вывести имена точек).


Определить и вывести на экран, какие 3 из заданных точек лежат на одной прямой (пользуясь словарем точек, вывести имена точек).

Вариант N 41

Даны точки: H (-3, -1, 1), K (0, 1, -4), L (0, -4, -4), P (1, -1, -1), Q (-23, -1, 11).
Составить словарь с ключами - точками (например, Point3D(3, 2, 1)) и значениями - именами точек (H, L и т.д.).

 
Найти и вывести на экран


расстояние от точек K и Q до плоскости HLP,

 
уравнение плоскости HLP,

 
параметрические уравнения прямой KQ.


Определить и вывести на экран, какие 4 из заданных точек лежат в одной плоскости (пользуясь словарем точек, вывести имена точек).


Определить и вывести на экран, какие 3 из заданных точек лежат на одной прямой (пользуясь словарем точек, вывести имена точек).

Вариант N 42

Даны точки: J (15, 12, 15), M (3, 2, 5), N (-3, 11, 4), R (5, 0, 3), T (-3, -3, 0).
Составить словарь с ключами - точками (например, Point3D(3, 2, 1)) и значениями - именами точек (M, R и т.д.).

 
Найти и вывести на экран


расстояние от точек N и J до плоскости MRT,

 
уравнение плоскости MRT,

 
параметрические уравнения прямой NJ.


Определить и вывести на экран, какие 4 из заданных точек лежат в одной плоскости (пользуясь словарем точек, вывести имена точек).


Определить и вывести на экран, какие 3 из заданных точек лежат на одной прямой (пользуясь словарем точек, вывести имена точек).

Вариант N 43

Даны точки: F (-1, 5, -1), K (0, -3, 4), M (3, -1, -1), N (-21, -19, -1), P (-1, -4, -1).
Составить словарь с ключами - точками (например, Point3D(3, 2, 1)) и значениями - именами точек (K, M и т.д.).

 
Найти и вывести на экран


расстояние от точек F и N до плоскости KMP,

 
уравнение плоскости KMP,

 
параметрические уравнения прямой FN.


Определить и вывести на экран, какие 4 из заданных точек лежат в одной плоскости (пользуясь словарем точек, вывести имена точек).


Определить и вывести на экран, какие 3 из заданных точек лежат на одной прямой (пользуясь словарем точек, вывести имена точек).

Вариант N 44

Даны точки: G (5, 3, -4), H (1, 3, 5), K (0, 1, -2), R (11, 1, 21), S (-4, 4, -3).
Составить словарь с ключами - точками (например, Point3D(3, 2, 1)) и значениями - именами точек (G, H и т.д.).

 
Найти и вывести на экран


расстояние от точек K и R до плоскости GHS,

 
уравнение плоскости GHS,

 
параметрические уравнения прямой KR.


Определить и вывести на экран, какие 4 из заданных точек лежат в одной плоскости (пользуясь словарем точек, вывести имена точек).


Определить и вывести на экран, какие 3 из заданных точек лежат на одной прямой (пользуясь словарем точек, вывести имена точек).

Вариант N 45

Даны точки: F (-1, 4, 2), H (-4, 1, -3), L (2, 0, 5), N (-9, 11, -18), P (-3, -1, 0).
Составить словарь с ключами - точками (например, Point3D(3, 2, 1)) и значениями - именами точек (H, L и т.д.).

 
Найти и вывести на экран


расстояние от точек F и N до плоскости HLP,

 
уравнение плоскости HLP,

 
параметрические уравнения прямой FN.


Определить и вывести на экран, какие 4 из заданных точек лежат в одной плоскости (пользуясь словарем точек, вывести имена точек).


Определить и вывести на экран, какие 3 из заданных точек лежат на одной прямой (пользуясь словарем точек, вывести имена точек).

Вариант N 46

Даны точки: F (0, 12, -15), H (5, -3, 5), K (2, -3, -2), L (-2, -3, -2), P (4, 0, 1).
Составить словарь с ключами - точками (например, Point3D(3, 2, 1)) и значениями - именами точек (H, L и т.д.).

 
Найти и вывести на экран


расстояние от точек K и F до плоскости HLP,

 
уравнение плоскости HLP,

 
параметрические уравнения прямой KF.


Определить и вывести на экран, какие 4 из заданных точек лежат в одной плоскости (пользуясь словарем точек, вывести имена точек).


Определить и вывести на экран, какие 3 из заданных точек лежат на одной прямой (пользуясь словарем точек, вывести имена точек).

Вариант N 47

Даны точки: G (5, -3, 3), K (-4, 5, 2), P (-1, 4, 3), R (-10, -2, -12), S (-4, 2, -2).
Составить словарь с ключами - точками (например, Point3D(3, 2, 1)) и значениями - именами точек (G, P и т.д.).

 
Найти и вывести на экран


расстояние от точек K и R до плоскости GPS,

 
уравнение плоскости GPS,

 
параметрические уравнения прямой KR.


Определить и вывести на экран, какие 4 из заданных точек лежат в одной плоскости (пользуясь словарем точек, вывести имена точек).


Определить и вывести на экран, какие 3 из заданных точек лежат на одной прямой (пользуясь словарем точек, вывести имена точек).

Вариант N 48

Даны точки: G (-4, 3, -1), K (4, -4, -1), M (5, -1, 2), P (-3, 2, 2), Q (-43, 17, 2).
Составить словарь с ключами - точками (например, Point3D(3, 2, 1)) и значениями - именами точек (G, M и т.д.).

 
Найти и вывести на экран


расстояние от точек K и Q до плоскости GMP,

 
уравнение плоскости GMP,

 
параметрические уравнения прямой KQ.


Определить и вывести на экран, какие 4 из заданных точек лежат в одной плоскости (пользуясь словарем точек, вывести имена точек).


Определить и вывести на экран, какие 3 из заданных точек лежат на одной прямой (пользуясь словарем точек, вывести имена точек).

Вариант N 49

Даны точки: F (2, -4, 1), G (-1, -4, 2), J (-7, 12, -6), K (-4, 4, -2), Q (1, -4, -4).
Составить словарь с ключами - точками (например, Point3D(3, 2, 1)) и значениями - именами точек (F, G и т.д.).

 
Найти и вывести на экран


расстояние от точек Q и J до плоскости FGK,

 
уравнение плоскости FGK,

 
параметрические уравнения прямой QJ.


Определить и вывести на экран, какие 4 из заданных точек лежат в одной плоскости (пользуясь словарем точек, вывести имена точек).


Определить и вывести на экран, какие 3 из заданных точек лежат на одной прямой (пользуясь словарем точек, вывести имена точек).

Вариант N 50

Даны точки: G (4, 2, -3), H (0, 5, -3), J (-8, 23, -11), L (1, -1, 4), T (4, -4, 1).
Составить словарь с ключами - точками (например, Point3D(3, 2, 1)) и значениями - именами точек (G, H и т.д.).

 
Найти и вывести на экран


расстояние от точек L и J до плоскости GHT,

 
уравнение плоскости GHT,

 
параметрические уравнения прямой LJ.


Определить и вывести на экран, какие 4 из заданных точек лежат в одной плоскости (пользуясь словарем точек, вывести имена точек).


Определить и вывести на экран, какие 3 из заданных точек лежат на одной прямой (пользуясь словарем точек, вывести имена точек).

Вариант N 51

Даны точки: F (-4, 4, -3), M (1, 1, 3), P (2, 3, 5), R (-23, 3, -25), S (-3, 3, -1).
Составить словарь с ключами - точками (например, Point3D(3, 2, 1)) и значениями - именами точек (M, P и т.д.).

 
Найти и вывести на экран


расстояние от точек F и R до плоскости MPS,

 
уравнение плоскости MPS,

 
параметрические уравнения прямой FR.


Определить и вывести на экран, какие 4 из заданных точек лежат в одной плоскости (пользуясь словарем точек, вывести имена точек).


Определить и вывести на экран, какие 3 из заданных точек лежат на одной прямой (пользуясь словарем точек, вывести имена точек).

Вариант N 52

Даны точки: G (5, 4, 1), L (0, -3, -3), M (3, 5, 3), Q (3, 2, 3), T (3, 4, 3).
Составить словарь с ключами - точками (например, Point3D(3, 2, 1)) и значениями - именами точек (L, M и т.д.).

 
Найти и вывести на экран


расстояние от точек G и Q до плоскости LMT,

 
уравнение плоскости LMT,

 
параметрические уравнения прямой GQ.


Определить и вывести на экран, какие 4 из заданных точек лежат в одной плоскости (пользуясь словарем точек, вывести имена точек).


Определить и вывести на экран, какие 3 из заданных точек лежат на одной прямой (пользуясь словарем точек, вывести имена точек).

Вариант N 53

Даны точки: F (-16, -9, -17), H (-4, -1, -3), K (4, -4, 1), M (2, 3, 4), Q (3, 5, 5).
Составить словарь с ключами - точками (например, Point3D(3, 2, 1)) и значениями - именами точек (H, K и т.д.).

 
Найти и вывести на экран


расстояние от точек Q и F до плоскости HKM,

 
уравнение плоскости HKM,

 
параметрические уравнения прямой QF.


Определить и вывести на экран, какие 4 из заданных точек лежат в одной плоскости (пользуясь словарем точек, вывести имена точек).


Определить и вывести на экран, какие 3 из заданных точек лежат на одной прямой (пользуясь словарем точек, вывести имена точек).

Вариант N 54

Даны точки: L (3, -2, -4), M (1, -4, 2), P (1, -1, -1), Q (3, 0, -2), R (1, -10, 8).
Составить словарь с ключами - точками (например, Point3D(3, 2, 1)) и значениями - именами точек (L, M и т.д.).

 
Найти и вывести на экран


расстояние от точек Q и R до плоскости LMP,

 
уравнение плоскости LMP,

 
параметрические уравнения прямой QR.


Определить и вывести на экран, какие 4 из заданных точек лежат в одной плоскости (пользуясь словарем точек, вывести имена точек).


Определить и вывести на экран, какие 3 из заданных точек лежат на одной прямой (пользуясь словарем точек, вывести имена точек).

Вариант N 55

Даны точки: F (-1, 2, -2), H (5, -4, 1), J (5, 14, 3), L (5, 5, 2), Q (5, 1, -4).
Составить словарь с ключами - точками (например, Point3D(3, 2, 1)) и значениями - именами точек (F, H и т.д.).

 
Найти и вывести на экран


расстояние от точек Q и J до плоскости FHL,

 
уравнение плоскости FHL,

 
параметрические уравнения прямой QJ.


Определить и вывести на экран, какие 4 из заданных точек лежат в одной плоскости (пользуясь словарем точек, вывести имена точек).


Определить и вывести на экран, какие 3 из заданных точек лежат на одной прямой (пользуясь словарем точек, вывести имена точек).

Вариант N 56

Даны точки: F (5, 2, -4), L (4, -4, 3), N (-16, 31, -2), S (5, 0, 0), T (0, 3, 2).
Составить словарь с ключами - точками (например, Point3D(3, 2, 1)) и значениями - именами точек (L, S и т.д.).

 
Найти и вывести на экран


расстояние от точек F и N до плоскости LST,

 
уравнение плоскости LST,

 
параметрические уравнения прямой FN.


Определить и вывести на экран, какие 4 из заданных точек лежат в одной плоскости (пользуясь словарем точек, вывести имена точек).


Определить и вывести на экран, какие 3 из заданных точек лежат на одной прямой (пользуясь словарем точек, вывести имена точек).

Вариант N 57

Даны точки: F (1, -4, 1), H (4, -4, 0), P (2, 4, 5), R (14, -44, -25), S (1, 0, 3).
Составить словарь с ключами - точками (например, Point3D(3, 2, 1)) и значениями - именами точек (H, P и т.д.).

 
Найти и вывести на экран


расстояние от точек F и R до плоскости HPS,

 
уравнение плоскости HPS,

 
параметрические уравнения прямой FR.


Определить и вывести на экран, какие 4 из заданных точек лежат в одной плоскости (пользуясь словарем точек, вывести имена точек).


Определить и вывести на экран, какие 3 из заданных точек лежат на одной прямой (пользуясь словарем точек, вывести имена точек).

Вариант N 58

Даны точки: H (-2, -4, -1), P (1, 3, -3), Q (-1, 0, -4), R (13, 31, -11), S (4, 5, 5).
Составить словарь с ключами - точками (например, Point3D(3, 2, 1)) и значениями - именами точек (H, P и т.д.).

 
Найти и вывести на экран


расстояние от точек Q и R до плоскости HPS,

 
уравнение плоскости HPS,

 
параметрические уравнения прямой QR.


Определить и вывести на экран, какие 4 из заданных точек лежат в одной плоскости (пользуясь словарем точек, вывести имена точек).


Определить и вывести на экран, какие 3 из заданных точек лежат на одной прямой (пользуясь словарем точек, вывести имена точек).

Вариант N 59

Даны точки: F (-1, 2, 3), J (-1, -2, 8), L (-1, -2, 2), M (-1, -2, 3), R (-2, -2, -2).
Составить словарь с ключами - точками (например, Point3D(3, 2, 1)) и значениями - именами точек (F, L и т.д.).

 
Найти и вывести на экран


расстояние от точек R и J до плоскости FLM,

 
уравнение плоскости FLM,

 
параметрические уравнения прямой RJ.


Определить и вывести на экран, какие 4 из заданных точек лежат в одной плоскости (пользуясь словарем точек, вывести имена точек).


Определить и вывести на экран, какие 3 из заданных точек лежат на одной прямой (пользуясь словарем точек, вывести имена точек).

Вариант N 60

Даны точки: H (-2, 1, 5), J (10, 9, -31), N (4, 4, -2), P (1, 3, -4), R (0, 0, -4).
Составить словарь с ключами - точками (например, Point3D(3, 2, 1)) и значениями - именами точек (H, P и т.д.).

 
Найти и вывести на экран


расстояние от точек N и J до плоскости HPR,

 
уравнение плоскости HPR,

 
параметрические уравнения прямой NJ.


Определить и вывести на экран, какие 4 из заданных точек лежат в одной плоскости (пользуясь словарем точек, вывести имена точек).


Определить и вывести на экран, какие 3 из заданных точек лежат на одной прямой (пользуясь словарем точек, вывести имена точек).

Вариант N 61

Даны точки: F (-4, 5, 2), L (2, -4, -2), N (-1, -16, -2), Q (-4, -4, 3), S (3, 0, -2).
Составить словарь с ключами - точками (например, Point3D(3, 2, 1)) и значениями - именами точек (L, Q и т.д.).

 
Найти и вывести на экран


расстояние от точек F и N до плоскости LQS,

 
уравнение плоскости LQS,

 
параметрические уравнения прямой FN.


Определить и вывести на экран, какие 4 из заданных точек лежат в одной плоскости (пользуясь словарем точек, вывести имена точек).


Определить и вывести на экран, какие 3 из заданных точек лежат на одной прямой (пользуясь словарем точек, вывести имена точек).

Вариант N 62

Даны точки: H (-3, 3, -3), L (5, -2, 3), P (1, -1, 3), R (9, -9, 15), S (2, 5, 0).
Составить словарь с ключами - точками (например, Point3D(3, 2, 1)) и значениями - именами точек (H, P и т.д.).

 
Найти и вывести на экран


расстояние от точек L и R до плоскости HPS,

 
уравнение плоскости HPS,

 
параметрические уравнения прямой LR.


Определить и вывести на экран, какие 4 из заданных точек лежат в одной плоскости (пользуясь словарем точек, вывести имена точек).


Определить и вывести на экран, какие 3 из заданных точек лежат на одной прямой (пользуясь словарем точек, вывести имена точек).

Вариант N 63

Даны точки: F (-4, 4, -1), J (-3, -1, 5), K (-1, 1, 5), N (1, 0, -3), P (1, 3, 5).
Составить словарь с ключами - точками (например, Point3D(3, 2, 1)) и значениями - именами точек (F, K и т.д.).

 
Найти и вывести на экран


расстояние от точек N и J до плоскости FKP,

 
уравнение плоскости FKP,

 
параметрические уравнения прямой NJ.


Определить и вывести на экран, какие 4 из заданных точек лежат в одной плоскости (пользуясь словарем точек, вывести имена точек).


Определить и вывести на экран, какие 3 из заданных точек лежат на одной прямой (пользуясь словарем точек, вывести имена точек).

Вариант N 64

Даны точки: H (0, 0, 2), K (3, -1, 0), L (5, -3, 0), P (4, 4, -2), R (-12, -12, 14).
Составить словарь с ключами - точками (например, Point3D(3, 2, 1)) и значениями - именами точек (H, L и т.д.).

 
Найти и вывести на экран


расстояние от точек K и R до плоскости HLP,

 
уравнение плоскости HLP,

 
параметрические уравнения прямой KR.


Определить и вывести на экран, какие 4 из заданных точек лежат в одной плоскости (пользуясь словарем точек, вывести имена точек).


Определить и вывести на экран, какие 3 из заданных точек лежат на одной прямой (пользуясь словарем точек, вывести имена точек).

Вариант N 65

Даны точки: K (37, -31, -23), L (-3, 5, 5), M (-2, 2, -1), P (5, -3, -3), T (-3, 4, 2).
Составить словарь с ключами - точками (например, Point3D(3, 2, 1)) и значениями - именами точек (M, P и т.д.).

 
Найти и вывести на экран


расстояние от точек L и K до плоскости MPT,

 
уравнение плоскости MPT,

 
параметрические уравнения прямой LK.


Определить и вывести на экран, какие 4 из заданных точек лежат в одной плоскости (пользуясь словарем точек, вывести имена точек).


Определить и вывести на экран, какие 3 из заданных точек лежат на одной прямой (пользуясь словарем точек, вывести имена точек).

Вариант N 66

Даны точки: J (-6, 3, 11), M (0, 3, 1), N (-2, 5, -4), R (-3, 0, 0), S (3, 3, -4).
Составить словарь с ключами - точками (например, Point3D(3, 2, 1)) и значениями - именами точек (M, R и т.д.).

 
Найти и вывести на экран


расстояние от точек N и J до плоскости MRS,

 
уравнение плоскости MRS,

 
параметрические уравнения прямой NJ.


Определить и вывести на экран, какие 4 из заданных точек лежат в одной плоскости (пользуясь словарем точек, вывести имена точек).


Определить и вывести на экран, какие 3 из заданных точек лежат на одной прямой (пользуясь словарем точек, вывести имена точек).

Вариант N 67

Даны точки: F (0, -3, 2), J (24, 14, -14), L (0, -4, 4), Q (0, 2, -2), T (4, -1, 1).
Составить словарь с ключами - точками (например, Point3D(3, 2, 1)) и значениями - именами точек (L, Q и т.д.).

 
Найти и вывести на экран


расстояние от точек F и J до плоскости LQT,

 
уравнение плоскости LQT,

 
параметрические уравнения прямой FJ.


Определить и вывести на экран, какие 4 из заданных точек лежат в одной плоскости (пользуясь словарем точек, вывести имена точек).


Определить и вывести на экран, какие 3 из заданных точек лежат на одной прямой (пользуясь словарем точек, вывести имена точек).

Вариант N 68

Даны точки: F (-1, -1, 1), G (2, 3, 4), Q (5, -3, -1), R (-6, -1, -16), S (0, 2, -1).
Составить словарь с ключами - точками (например, Point3D(3, 2, 1)) и значениями - именами точек (G, Q и т.д.).

 
Найти и вывести на экран


расстояние от точек F и R до плоскости GQS,

 
уравнение плоскости GQS,

 
параметрические уравнения прямой FR.


Определить и вывести на экран, какие 4 из заданных точек лежат в одной плоскости (пользуясь словарем точек, вывести имена точек).


Определить и вывести на экран, какие 3 из заданных точек лежат на одной прямой (пользуясь словарем точек, вывести имена точек).

Вариант N 69

Даны точки: J (44, 21, -31), Q (0, 1, -3), R (3, 3, 0), S (-4, -3, 5), T (4, 1, -1).
Составить словарь с ключами - точками (например, Point3D(3, 2, 1)) и значениями - именами точек (Q, S и т.д.).

 
Найти и вывести на экран


расстояние от точек R и J до плоскости QST,

 
уравнение плоскости QST,

 
параметрические уравнения прямой RJ.


Определить и вывести на экран, какие 4 из заданных точек лежат в одной плоскости (пользуясь словарем точек, вывести имена точек).


Определить и вывести на экран, какие 3 из заданных точек лежат на одной прямой (пользуясь словарем точек, вывести имена точек).

Вариант N 70

Даны точки: F (-3, 4, 3), L (-4, 3, -2), P (5, 1, -1), Q (3, -1, 0), R (50, -9, 4).
Составить словарь с ключами - точками (например, Point3D(3, 2, 1)) и значениями - именами точек (L, P и т.д.).

 
Найти и вывести на экран


расстояние от точек F и R до плоскости LPQ,

 
уравнение плоскости LPQ,

 
параметрические уравнения прямой FR.


Определить и вывести на экран, какие 4 из заданных точек лежат в одной плоскости (пользуясь словарем точек, вывести имена точек).


Определить и вывести на экран, какие 3 из заданных точек лежат на одной прямой (пользуясь словарем точек, вывести имена точек).

Вариант N 71

Даны точки: F (-1, 4, 3), G (-2, -3, 2), J (2, -7, 26), N (-4, 1, -1), P (-3, -2, -4).
Составить словарь с ключами - точками (например, Point3D(3, 2, 1)) и значениями - именами точек (F, G и т.д.).

 
Найти и вывести на экран


расстояние от точек N и J до плоскости FGP,

 
уравнение плоскости FGP,

 
параметрические уравнения прямой NJ.


Определить и вывести на экран, какие 4 из заданных точек лежат в одной плоскости (пользуясь словарем точек, вывести имена точек).


Определить и вывести на экран, какие 3 из заданных точек лежат на одной прямой (пользуясь словарем точек, вывести имена точек).

Вариант N 72

Даны точки: G (3, -3, -3), K (2, 2, 1), M (-4, 4, -4), P (-3, 5, 0), R (0, 8, 12).
Составить словарь с ключами - точками (например, Point3D(3, 2, 1)) и значениями - именами точек (G, M и т.д.).

 
Найти и вывести на экран


расстояние от точек K и R до плоскости GMP,

 
уравнение плоскости GMP,

 
параметрические уравнения прямой KR.


Определить и вывести на экран, какие 4 из заданных точек лежат в одной плоскости (пользуясь словарем точек, вывести имена точек).


Определить и вывести на экран, какие 3 из заданных точек лежат на одной прямой (пользуясь словарем точек, вывести имена точек).

Вариант N 73

Даны точки: J (-13, -28, -4), N (-3, -4, 0), P (-1, -4, -4), Q (2, 2, -4), R (-4, 0, -1).
Составить словарь с ключами - точками (например, Point3D(3, 2, 1)) и значениями - именами точек (P, Q и т.д.).

 
Найти и вывести на экран


расстояние от точек N и J до плоскости PQR,

 
уравнение плоскости PQR,

 
параметрические уравнения прямой NJ.


Определить и вывести на экран, какие 4 из заданных точек лежат в одной плоскости (пользуясь словарем точек, вывести имена точек).


Определить и вывести на экран, какие 3 из заданных точек лежат на одной прямой (пользуясь словарем точек, вывести имена точек).

Вариант N 74

Даны точки: F (-1, 14, -1), L (-2, 4, 1), P (5, 2, 1), Q (7, 0, 1), S (4, 3, 1).
Составить словарь с ключами - точками (например, Point3D(3, 2, 1)) и значениями - именами точек (L, P и т.д.).

 
Найти и вывести на экран


расстояние от точек F и Q до плоскости LPS,

 
уравнение плоскости LPS,

 
параметрические уравнения прямой FQ.


Определить и вывести на экран, какие 4 из заданных точек лежат в одной плоскости (пользуясь словарем точек, вывести имена точек).


Определить и вывести на экран, какие 3 из заданных точек лежат на одной прямой (пользуясь словарем точек, вывести имена точек).

Вариант N 75

Даны точки: K (1, 1, 4), N (5, 9, 8), Q (-1, -3, 5), R (-4, 5, 0), S (0, -1, 3).
Составить словарь с ключами - точками (например, Point3D(3, 2, 1)) и значениями - именами точек (K, Q и т.д.).

 
Найти и вывести на экран


расстояние от точек R и N до плоскости KQS,

 
уравнение плоскости KQS,

 
параметрические уравнения прямой RN.


Определить и вывести на экран, какие 4 из заданных точек лежат в одной плоскости (пользуясь словарем точек, вывести имена точек).


Определить и вывести на экран, какие 3 из заданных точек лежат на одной прямой (пользуясь словарем точек, вывести имена точек).

Вариант N 76

Даны точки: H (3, -2, 1), J (15, -17, 16), K (1, 3, 2), L (-1, -4, -2), P (-1, 3, -4).
Составить словарь с ключами - точками (например, Point3D(3, 2, 1)) и значениями - именами точек (H, L и т.д.).

 
Найти и вывести на экран


расстояние от точек K и J до плоскости HLP,

 
уравнение плоскости HLP,

 
параметрические уравнения прямой KJ.


Определить и вывести на экран, какие 4 из заданных точек лежат в одной плоскости (пользуясь словарем точек, вывести имена точек).


Определить и вывести на экран, какие 3 из заданных точек лежат на одной прямой (пользуясь словарем точек, вывести имена точек).

Вариант N 77

Даны точки: F (4, 5, 4), L (-1, 4, -2), N (-19, 4, -14), Q (3, 5, 4), S (5, 4, 2).
Составить словарь с ключами - точками (например, Point3D(3, 2, 1)) и значениями - именами точек (F, L и т.д.).

 
Найти и вывести на экран


расстояние от точек Q и N до плоскости FLS,

 
уравнение плоскости FLS,

 
параметрические уравнения прямой QN.


Определить и вывести на экран, какие 4 из заданных точек лежат в одной плоскости (пользуясь словарем точек, вывести имена точек).


Определить и вывести на экран, какие 3 из заданных точек лежат на одной прямой (пользуясь словарем точек, вывести имена точек).

Вариант N 78

Даны точки: F (-40, 9, -19), L (1, -4, -3), M (-2, -3, -2), P (-4, 1, -3), S (5, -1, 1).
Составить словарь с ключами - точками (например, Point3D(3, 2, 1)) и значениями - именами точек (M, P и т.д.).

 
Найти и вывести на экран


расстояние от точек L и F до плоскости MPS,

 
уравнение плоскости MPS,

 
параметрические уравнения прямой LF.


Определить и вывести на экран, какие 4 из заданных точек лежат в одной плоскости (пользуясь словарем точек, вывести имена точек).


Определить и вывести на экран, какие 3 из заданных точек лежат на одной прямой (пользуясь словарем точек, вывести имена точек).

Вариант N 79

Даны точки: J (-20, -4, 2), L (4, 2, 2), M (0, 1, 2), Q (-2, 4, 4), R (1, 1, -4).
Составить словарь с ключами - точками (например, Point3D(3, 2, 1)) и значениями - именами точек (L, M и т.д.).

 
Найти и вывести на экран


расстояние от точек R и J до плоскости LMQ,

 
уравнение плоскости LMQ,

 
параметрические уравнения прямой RJ.


Определить и вывести на экран, какие 4 из заданных точек лежат в одной плоскости (пользуясь словарем точек, вывести имена точек).


Определить и вывести на экран, какие 3 из заданных точек лежат на одной прямой (пользуясь словарем точек, вывести имена точек).

Вариант N 80

Даны точки: F (-3, -3, 4), H (2, 3, 5), J (-13, -15, 2), N (1, 4, 1), R (3, 5, 5).
Составить словарь с ключами - точками (например, Point3D(3, 2, 1)) и значениями - именами точек (F, H и т.д.).

 
Найти и вывести на экран


расстояние от точек N и J до плоскости FHR,

 
уравнение плоскости FHR,

 
параметрические уравнения прямой NJ.


Определить и вывести на экран, какие 4 из заданных точек лежат в одной плоскости (пользуясь словарем точек, вывести имена точек).


Определить и вывести на экран, какие 3 из заданных точек лежат на одной прямой (пользуясь словарем точек, вывести имена точек).

Вариант N 81

Даны точки: J (-15, -14, -30), K (4, 2, -2), L (0, 1, 5), N (-1, 0, 5), T (-3, -2, -2).
Составить словарь с ключами - точками (например, Point3D(3, 2, 1)) и значениями - именами точек (K, L и т.д.).

 
Найти и вывести на экран


расстояние от точек N и J до плоскости KLT,

 
уравнение плоскости KLT,

 
параметрические уравнения прямой NJ.


Определить и вывести на экран, какие 4 из заданных точек лежат в одной плоскости (пользуясь словарем точек, вывести имена точек).


Определить и вывести на экран, какие 3 из заданных точек лежат на одной прямой (пользуясь словарем точек, вывести имена точек).

Вариант N 82

Даны точки: F (0, -1, 3), G (3, 3, 1), H (-3, -4, -3), J (13, 4, 25), M (1, -2, 4).
Составить словарь с ключами - точками (например, Point3D(3, 2, 1)) и значениями - именами точек (G, H и т.д.).

 
Найти и вывести на экран


расстояние от точек F и J до плоскости GHM,

 
уравнение плоскости GHM,

 
параметрические уравнения прямой FJ.


Определить и вывести на экран, какие 4 из заданных точек лежат в одной плоскости (пользуясь словарем точек, вывести имена точек).


Определить и вывести на экран, какие 3 из заданных точек лежат на одной прямой (пользуясь словарем точек, вывести имена точек).

Вариант N 83

Даны точки: K (1, 3, -4), L (-2, 1, -2), M (-1, -1, -4), P (3, 3, -1), Q (-17, -5, -5).
Составить словарь с ключами - точками (например, Point3D(3, 2, 1)) и значениями - именами точек (L, M и т.д.).

 
Найти и вывести на экран


расстояние от точек K и Q до плоскости LMP,

 
уравнение плоскости LMP,

 
параметрические уравнения прямой KQ.


Определить и вывести на экран, какие 4 из заданных точек лежат в одной плоскости (пользуясь словарем точек, вывести имена точек).


Определить и вывести на экран, какие 3 из заданных точек лежат на одной прямой (пользуясь словарем точек, вывести имена точек).

Вариант N 84

Даны точки: H (1, -1, 1), J (3, 3, 3), L (1, -4, 4), P (2, 1, 2), R (5, 13, 4).
Составить словарь с ключами - точками (например, Point3D(3, 2, 1)) и значениями - именами точек (H, L и т.д.).

 
Найти и вывести на экран


расстояние от точек R и J до плоскости HLP,

 
уравнение плоскости HLP,

 
параметрические уравнения прямой RJ.


Определить и вывести на экран, какие 4 из заданных точек лежат в одной плоскости (пользуясь словарем точек, вывести имена точек).


Определить и вывести на экран, какие 3 из заданных точек лежат на одной прямой (пользуясь словарем точек, вывести имена точек).

Вариант N 85

Даны точки: G (-3, 2, 4), J (-10, 1, 12), M (-1, -2, 4), R (5, -3, 5), S (4, 3, -4).
Составить словарь с ключами - точками (например, Point3D(3, 2, 1)) и значениями - именами точек (G, M и т.д.).

 
Найти и вывести на экран


расстояние от точек R и J до плоскости GMS,

 
уравнение плоскости GMS,

 
параметрические уравнения прямой RJ.


Определить и вывести на экран, какие 4 из заданных точек лежат в одной плоскости (пользуясь словарем точек, вывести имена точек).


Определить и вывести на экран, какие 3 из заданных точек лежат на одной прямой (пользуясь словарем точек, вывести имена точек).

Вариант N 86

Даны точки: F (2, 3, 0), J (18, 3, 0), M (-2, 3, 0), N (-2, -2, 5), Q (4, -4, 0).
Составить словарь с ключами - точками (например, Point3D(3, 2, 1)) и значениями - именами точек (F, M и т.д.).

 
Найти и вывести на экран


расстояние от точек N и J до плоскости FMQ,

 
уравнение плоскости FMQ,

 
параметрические уравнения прямой NJ.


Определить и вывести на экран, какие 4 из заданных точек лежат в одной плоскости (пользуясь словарем точек, вывести имена точек).


Определить и вывести на экран, какие 3 из заданных точек лежат на одной прямой (пользуясь словарем точек, вывести имена точек).

Вариант N 87

Даны точки: H (5, -2, 0), K (-2, -2, 1), N (-20, 33, 15), R (-4, 1, 4), T (0, 5, 3).
Составить словарь с ключами - точками (например, Point3D(3, 2, 1)) и значениями - именами точек (H, K и т.д.).

 
Найти и вывести на экран


расстояние от точек R и N до плоскости HKT,

 
уравнение плоскости HKT,

 
параметрические уравнения прямой RN.


Определить и вывести на экран, какие 4 из заданных точек лежат в одной плоскости (пользуясь словарем точек, вывести имена точек).


Определить и вывести на экран, какие 3 из заданных точек лежат на одной прямой (пользуясь словарем точек, вывести имена точек).

Вариант N 88

Даны точки: H (0, -1, 0), K (-4, 0, 3), N (6, 5, -8), S (-3, 2, 2), T (3, 2, -4).
Составить словарь с ключами - точками (например, Point3D(3, 2, 1)) и значениями - именами точек (H, S и т.д.).

 
Найти и вывести на экран


расстояние от точек K и N до плоскости HST,

 
уравнение плоскости HST,

 
параметрические уравнения прямой KN.


Определить и вывести на экран, какие 4 из заданных точек лежат в одной плоскости (пользуясь словарем точек, вывести имена точек).


Определить и вывести на экран, какие 3 из заданных точек лежат на одной прямой (пользуясь словарем точек, вывести имена точек).

Вариант N 89

Даны точки: F (9, -1, -15), L (0, 4, -2), M (-3, 2, 0), P (0, 2, 3), S (3, 1, -3).
Составить словарь с ключами - точками (например, Point3D(3, 2, 1)) и значениями - именами точек (M, P и т.д.).

 
Найти и вывести на экран


расстояние от точек L и F до плоскости MPS,

 
уравнение плоскости MPS,

 
параметрические уравнения прямой LF.


Определить и вывести на экран, какие 4 из заданных точек лежат в одной плоскости (пользуясь словарем точек, вывести имена точек).


Определить и вывести на экран, какие 3 из заданных точек лежат на одной прямой (пользуясь словарем точек, вывести имена точек).

Вариант N 90

Даны точки: H (3, 4, -2), L (6, 4, -2), P (4, 4, -2), S (-4, 5, 2), T (-3, 2, -3).
Составить словарь с ключами - точками (например, Point3D(3, 2, 1)) и значениями - именами точек (H, P и т.д.).

 
Найти и вывести на экран


расстояние от точек T и L до плоскости HPS,

 
уравнение плоскости HPS,

 
параметрические уравнения прямой TL.


Определить и вывести на экран, какие 4 из заданных точек лежат в одной плоскости (пользуясь словарем точек, вывести имена точек).


Определить и вывести на экран, какие 3 из заданных точек лежат на одной прямой (пользуясь словарем точек, вывести имена точек).

Вариант N 91

Даны точки: G (4, -3, 3), H (1, -3, -3), K (1, -1, -3), Q (-14, -8, 27), S (-2, -4, 3).
Составить словарь с ключами - точками (например, Point3D(3, 2, 1)) и значениями - именами точек (G, H и т.д.).

 
Найти и вывести на экран


расстояние от точек K и Q до плоскости GHS,

 
уравнение плоскости GHS,

 
параметрические уравнения прямой KQ.


Определить и вывести на экран, какие 4 из заданных точек лежат в одной плоскости (пользуясь словарем точек, вывести имена точек).


Определить и вывести на экран, какие 3 из заданных точек лежат на одной прямой (пользуясь словарем точек, вывести имена точек).

Вариант N 92

Даны точки: F (2, 4, -1), J (-2, 18, -1), N (3, -1, -4), Q (4, 4, 4), S (4, -3, -1).
Составить словарь с ключами - точками (например, Point3D(3, 2, 1)) и значениями - именами точек (F, Q и т.д.).

 
Найти и вывести на экран


расстояние от точек N и J до плоскости FQS,

 
уравнение плоскости FQS,

 
параметрические уравнения прямой NJ.


Определить и вывести на экран, какие 4 из заданных точек лежат в одной плоскости (пользуясь словарем точек, вывести имена точек).


Определить и вывести на экран, какие 3 из заданных точек лежат на одной прямой (пользуясь словарем точек, вывести имена точек).

Вариант N 93

Даны точки: H (5, -4, 5), J (-3, 6, -11), K (-1, 0, 5), P (1, 1, -3), T (4, 3, -2).
Составить словарь с ключами - точками (например, Point3D(3, 2, 1)) и значениями - именами точек (H, P и т.д.).

 
Найти и вывести на экран


расстояние от точек K и J до плоскости HPT,

 
уравнение плоскости HPT,

 
параметрические уравнения прямой KJ.


Определить и вывести на экран, какие 4 из заданных точек лежат в одной плоскости (пользуясь словарем точек, вывести имена точек).


Определить и вывести на экран, какие 3 из заданных точек лежат на одной прямой (пользуясь словарем точек, вывести имена точек).

Вариант N 94

Даны точки: F (-9, 5, -4), G (0, 2, -1), L (3, 1, 0), Q (3, -2, 3), R (4, 10, 3).
Составить словарь с ключами - точками (например, Point3D(3, 2, 1)) и значениями - именами точек (G, L и т.д.).

 
Найти и вывести на экран


расстояние от точек R и F до плоскости GLQ,

 
уравнение плоскости GLQ,

 
параметрические уравнения прямой RF.


Определить и вывести на экран, какие 4 из заданных точек лежат в одной плоскости (пользуясь словарем точек, вывести имена точек).


Определить и вывести на экран, какие 3 из заданных точек лежат на одной прямой (пользуясь словарем точек, вывести имена точек).

Вариант N 95

Даны точки: F (-31, 18, 29), G (0, 3, -1), H (5, -2, -3), M (5, 1, 0), S (-4, 3, 5).
Составить словарь с ключами - точками (например, Point3D(3, 2, 1)) и значениями - именами точек (H, M и т.д.).

 
Найти и вывести на экран


расстояние от точек G и F до плоскости HMS,

 
уравнение плоскости HMS,

 
параметрические уравнения прямой GF.


Определить и вывести на экран, какие 4 из заданных точек лежат в одной плоскости (пользуясь словарем точек, вывести имена точек).


Определить и вывести на экран, какие 3 из заданных точек лежат на одной прямой (пользуясь словарем точек, вывести имена точек).

Вариант N 96

Даны точки: H (5, -3, -2), K (-2, -4, 2), L (-2, -3, 5), M (-3, -3, 4), Q (45, -3, -32).
Составить словарь с ключами - точками (например, Point3D(3, 2, 1)) и значениями - именами точек (H, L и т.д.).

 
Найти и вывести на экран


расстояние от точек K и Q до плоскости HLM,

 
уравнение плоскости HLM,

 
параметрические уравнения прямой KQ.


Определить и вывести на экран, какие 4 из заданных точек лежат в одной плоскости (пользуясь словарем точек, вывести имена точек).


Определить и вывести на экран, какие 3 из заданных точек лежат на одной прямой (пользуясь словарем точек, вывести имена точек).

Вариант N 97

Даны точки: G (-3, 0, 1), J (-3, -7, -19), L (1, 5, 2), M (5, 5, 1), T (3, 2, -4).
Составить словарь с ключами - точками (например, Point3D(3, 2, 1)) и значениями - именами точек (G, M и т.д.).

 
Найти и вывести на экран


расстояние от точек L и J до плоскости GMT,

 
уравнение плоскости GMT,

 
параметрические уравнения прямой LJ.


Определить и вывести на экран, какие 4 из заданных точек лежат в одной плоскости (пользуясь словарем точек, вывести имена точек).


Определить и вывести на экран, какие 3 из заданных точек лежат на одной прямой (пользуясь словарем точек, вывести имена точек).

Вариант N 98

Даны точки: F (-2, 0, 5), G (-3, -4, 1), H (-3, 4, -4), J (1, -12, 8), S (-2, 0, -1).
Составить словарь с ключами - точками (например, Point3D(3, 2, 1)) и значениями - именами точек (G, H и т.д.).

 
Найти и вывести на экран


расстояние от точек F и J до плоскости GHS,

 
уравнение плоскости GHS,

 
параметрические уравнения прямой FJ.


Определить и вывести на экран, какие 4 из заданных точек лежат в одной плоскости (пользуясь словарем точек, вывести имена точек).


Определить и вывести на экран, какие 3 из заданных точек лежат на одной прямой (пользуясь словарем точек, вывести имена точек).

Вариант N 99

Даны точки: F (-2, -3, 0), K (-3, 5, 5), L (-2, -3, -4), P (-4, -3, 5), Q (-12, -3, 41).
Составить словарь с ключами - точками (например, Point3D(3, 2, 1)) и значениями - именами точек (K, L и т.д.).

 
Найти и вывести на экран


расстояние от точек F и Q до плоскости KLP,

 
уравнение плоскости KLP,

 
параметрические уравнения прямой FQ.


Определить и вывести на экран, какие 4 из заданных точек лежат в одной плоскости (пользуясь словарем точек, вывести имена точек).


Определить и вывести на экран, какие 3 из заданных точек лежат на одной прямой (пользуясь словарем точек, вывести имена точек).

Вариант N 100

Даны точки: P (4, -1, -4), Q (-2, -4, -4), R (24, 9, -44), S (0, -3, 4), T (2, -3, -4).
Составить словарь с ключами - точками (например, Point3D(3, 2, 1)) и значениями - именами точек (P, S и т.д.).

 
Найти и вывести на экран


расстояние от точек Q и R до плоскости PST,

 
уравнение плоскости PST,

 
параметрические уравнения прямой QR.


Определить и вывести на экран, какие 4 из заданных точек лежат в одной плоскости (пользуясь словарем точек, вывести имена точек).


Определить и вывести на экран, какие 3 из заданных точек лежат на одной прямой (пользуясь словарем точек, вывести имена точек).

Вариант N 101

Даны точки: G (5, -4, -2), J (-27, 37, 4), K (-3, 2, 5), M (3, -3, 4), S (-3, 5, 4).
Составить словарь с ключами - точками (например, Point3D(3, 2, 1)) и значениями - именами точек (G, M и т.д.).

 
Найти и вывести на экран


расстояние от точек K и J до плоскости GMS,

 
уравнение плоскости GMS,

 
параметрические уравнения прямой KJ.


Определить и вывести на экран, какие 4 из заданных точек лежат в одной плоскости (пользуясь словарем точек, вывести имена точек).


Определить и вывести на экран, какие 3 из заданных точек лежат на одной прямой (пользуясь словарем точек, вывести имена точек).

Вариант N 102

Даны точки: F (4, 0, -4), J (9, 4, -12), N (0, 3, 4), Q (3, 4, -3), S (1, 4, 0).
Составить словарь с ключами - точками (например, Point3D(3, 2, 1)) и значениями - именами точек (F, Q и т.д.).

 
Найти и вывести на экран


расстояние от точек N и J до плоскости FQS,

 
уравнение плоскости FQS,

 
параметрические уравнения прямой NJ.


Определить и вывести на экран, какие 4 из заданных точек лежат в одной плоскости (пользуясь словарем точек, вывести имена точек).


Определить и вывести на экран, какие 3 из заданных точек лежат на одной прямой (пользуясь словарем точек, вывести имена точек).

Вариант N 103

Даны точки: K (0, 1, -3), N (5, 21, 12), P (1, 5, 0), Q (4, -4, 2), R (4, 5, -3).
Составить словарь с ключами - точками (например, Point3D(3, 2, 1)) и значениями - именами точек (K, P и т.д.).

 
Найти и вывести на экран


расстояние от точек R и N до плоскости KPQ,

 
уравнение плоскости KPQ,

 
параметрические уравнения прямой RN.


Определить и вывести на экран, какие 4 из заданных точек лежат в одной плоскости (пользуясь словарем точек, вывести имена точек).


Определить и вывести на экран, какие 3 из заданных точек лежат на одной прямой (пользуясь словарем точек, вывести имена точек).

Вариант N 104

Даны точки: G (2, 3, -1), L (2, 1, -3), P (-1, 0, 2), R (2, -6, -16), T (0, -2, -4).
Составить словарь с ключами - точками (например, Point3D(3, 2, 1)) и значениями - именами точек (G, P и т.д.).

 
Найти и вывести на экран


расстояние от точек L и R до плоскости GPT,

 
уравнение плоскости GPT,

 
параметрические уравнения прямой LR.


Определить и вывести на экран, какие 4 из заданных точек лежат в одной плоскости (пользуясь словарем точек, вывести имена точек).


Определить и вывести на экран, какие 3 из заданных точек лежат на одной прямой (пользуясь словарем точек, вывести имена точек).

Вариант N 105

Даны точки: F (-2, -2, 3), G (2, -4, 0), H (4, 1, -3), J (-4, -19, 9), N (2, -2, 3).
Составить словарь с ключами - точками (например, Point3D(3, 2, 1)) и значениями - именами точек (F, G и т.д.).

 
Найти и вывести на экран


расстояние от точек N и J до плоскости FGH,

 
уравнение плоскости FGH,

 
параметрические уравнения прямой NJ.


Определить и вывести на экран, какие 4 из заданных точек лежат в одной плоскости (пользуясь словарем точек, вывести имена точек).


Определить и вывести на экран, какие 3 из заданных точек лежат на одной прямой (пользуясь словарем точек, вывести имена точек).

Вариант N 106

Даны точки: G (5, -1, 5), J (-38, -44, 38), M (-1, 2, -1), S (4, 4, -4), T (-3, -4, 3).
Составить словарь с ключами - точками (например, Point3D(3, 2, 1)) и значениями - именами точек (G, S и т.д.).

 
Найти и вывести на экран


расстояние от точек M и J до плоскости GST,

 
уравнение плоскости GST,

 
параметрические уравнения прямой MJ.


Определить и вывести на экран, какие 4 из заданных точек лежат в одной плоскости (пользуясь словарем точек, вывести имена точек).


Определить и вывести на экран, какие 3 из заданных точек лежат на одной прямой (пользуясь словарем точек, вывести имена точек).

Вариант N 107

Даны точки: H (-2, 5, -2), L (3, 0, -1), M (1, 1, -1), P (0, 4, -4), Q (4, 2, -8).
Составить словарь с ключами - точками (например, Point3D(3, 2, 1)) и значениями - именами точек (H, M и т.д.).

 
Найти и вывести на экран


расстояние от точек L и Q до плоскости HMP,

 
уравнение плоскости HMP,

 
параметрические уравнения прямой LQ.


Определить и вывести на экран, какие 4 из заданных точек лежат в одной плоскости (пользуясь словарем точек, вывести имена точек).


Определить и вывести на экран, какие 3 из заданных точек лежат на одной прямой (пользуясь словарем точек, вывести имена точек).

Вариант N 108

Даны точки: F (-2, -2, -2), G (1, -4, 2), K (-1, 4, -4), L (-2, 3, 4), N (-8, 17, 8).
Составить словарь с ключами - точками (например, Point3D(3, 2, 1)) и значениями - именами точек (G, K и т.д.).

 
Найти и вывести на экран


расстояние от точек F и N до плоскости GKL,

 
уравнение плоскости GKL,

 
параметрические уравнения прямой FN.


Определить и вывести на экран, какие 4 из заданных точек лежат в одной плоскости (пользуясь словарем точек, вывести имена точек).


Определить и вывести на экран, какие 3 из заданных точек лежат на одной прямой (пользуясь словарем точек, вывести имена точек).

Вариант N 109

Даны точки: F (-4, 4, 2), J (-32, 12, 18), N (1, 4, 2), P (3, 2, -2), Q (-4, 5, -2).
Составить словарь с ключами - точками (например, Point3D(3, 2, 1)) и значениями - именами точек (F, P и т.д.).

 
Найти и вывести на экран


расстояние от точек N и J до плоскости FPQ,

 
уравнение плоскости FPQ,

 
параметрические уравнения прямой NJ.


Определить и вывести на экран, какие 4 из заданных точек лежат в одной плоскости (пользуясь словарем точек, вывести имена точек).


Определить и вывести на экран, какие 3 из заданных точек лежат на одной прямой (пользуясь словарем точек, вывести имена точек).

Вариант N 110

Даны точки: H (2, 5, 2), J (22, 25, 22), L (2, 3, -2), M (0, -1, 1), P (-2, 1, -2).
Составить словарь с ключами - точками (например, Point3D(3, 2, 1)) и значениями - именами точек (H, M и т.д.).

 
Найти и вывести на экран


расстояние от точек L и J до плоскости HMP,

 
уравнение плоскости HMP,

 
параметрические уравнения прямой LJ.


Определить и вывести на экран, какие 4 из заданных точек лежат в одной плоскости (пользуясь словарем точек, вывести имена точек).


Определить и вывести на экран, какие 3 из заданных точек лежат на одной прямой (пользуясь словарем точек, вывести имена точек).

Вариант N 111

Даны точки: F (3, -3, 2), G (-1, 0, -2), N (-2, -3, -7), P (0, 3, 3), R (-4, -4, 1).
Составить словарь с ключами - точками (например, Point3D(3, 2, 1)) и значениями - именами точек (F, G и т.д.).

 
Найти и вывести на экран


расстояние от точек R и N до плоскости FGP,

 
уравнение плоскости FGP,

 
параметрические уравнения прямой RN.


Определить и вывести на экран, какие 4 из заданных точек лежат в одной плоскости (пользуясь словарем точек, вывести имена точек).


Определить и вывести на экран, какие 3 из заданных точек лежат на одной прямой (пользуясь словарем точек, вывести имена точек).

Вариант N 112

Даны точки: F (-4, -4, -1), H (3, 1, 0), J (15, -7, -2), K (2, -1, 2), S (-3, 5, 1).
Составить словарь с ключами - точками (например, Point3D(3, 2, 1)) и значениями - именами точек (H, K и т.д.).

 
Найти и вывести на экран


расстояние от точек F и J до плоскости HKS,

 
уравнение плоскости HKS,

 
параметрические уравнения прямой FJ.


Определить и вывести на экран, какие 4 из заданных точек лежат в одной плоскости (пользуясь словарем точек, вывести имена точек).


Определить и вывести на экран, какие 3 из заданных точек лежат на одной прямой (пользуясь словарем точек, вывести имена точек).

Вариант N 113

Даны точки: G (4, 5, 2), J (4, 7, 17), N (-1, -3, 3), S (4, 1, -1), T (4, 3, 5).
Составить словарь с ключами - точками (например, Point3D(3, 2, 1)) и значениями - именами точек (G, S и т.д.).

 
Найти и вывести на экран


расстояние от точек N и J до плоскости GST,

 
уравнение плоскости GST,

 
параметрические уравнения прямой NJ.


Определить и вывести на экран, какие 4 из заданных точек лежат в одной плоскости (пользуясь словарем точек, вывести имена точек).


Определить и вывести на экран, какие 3 из заданных точек лежат на одной прямой (пользуясь словарем точек, вывести имена точек).

Вариант N 114

Даны точки: F (3, 4, 4), G (3, 5, 1), K (-4, 5, 3), M (-2, -1, 1), R (18, 23, 1).
Составить словарь с ключами - точками (например, Point3D(3, 2, 1)) и значениями - именами точек (G, K и т.д.).

 
Найти и вывести на экран


расстояние от точек F и R до плоскости GKM,

 
уравнение плоскости GKM,

 
параметрические уравнения прямой FR.


Определить и вывести на экран, какие 4 из заданных точек лежат в одной плоскости (пользуясь словарем точек, вывести имена точек).


Определить и вывести на экран, какие 3 из заданных точек лежат на одной прямой (пользуясь словарем точек, вывести имена точек).

Вариант N 115

Даны точки: J (-7, -1, 13), K (2, -1, 1), N (1, -4, 5), R (5, -3, 5), S (-1, -1, 5).
Составить словарь с ключами - точками (например, Point3D(3, 2, 1)) и значениями - именами точек (K, R и т.д.).

 
Найти и вывести на экран


расстояние от точек N и J до плоскости KRS,

 
уравнение плоскости KRS,

 
параметрические уравнения прямой NJ.


Определить и вывести на экран, какие 4 из заданных точек лежат в одной плоскости (пользуясь словарем точек, вывести имена точек).


Определить и вывести на экран, какие 3 из заданных точек лежат на одной прямой (пользуясь словарем точек, вывести имена точек).

Вариант N 116

Даны точки: F (0, 0, 5), G (1, 0, 3), H (1, 3, -1), J (1, -9, 15), R (4, -4, 0).
Составить словарь с ключами - точками (например, Point3D(3, 2, 1)) и значениями - именами точек (F, G и т.д.).

 
Найти и вывести на экран


расстояние от точек R и J до плоскости FGH,

 
уравнение плоскости FGH,

 
параметрические уравнения прямой RJ.


Определить и вывести на экран, какие 4 из заданных точек лежат в одной плоскости (пользуясь словарем точек, вывести имена точек).


Определить и вывести на экран, какие 3 из заданных точек лежат на одной прямой (пользуясь словарем точек, вывести имена точек).

Вариант N 117

Даны точки: H (2, -4, 5), M (-4, 1, 5), P (5, 5, -4), R (17, 41, -40), S (-4, -3, 2).
Составить словарь с ключами - точками (например, Point3D(3, 2, 1)) и значениями - именами точек (H, P и т.д.).

 
Найти и вывести на экран


расстояние от точек M и R до плоскости HPS,

 
уравнение плоскости HPS,

 
параметрические уравнения прямой MR.


Определить и вывести на экран, какие 4 из заданных точек лежат в одной плоскости (пользуясь словарем точек, вывести имена точек).


Определить и вывести на экран, какие 3 из заданных точек лежат на одной прямой (пользуясь словарем точек, вывести имена точек).

Вариант N 118

Даны точки: G (-3, -2, 3), H (1, -2, -2), J (-19, -2, 23), K (3, -3, 5), R (-3, 0, -2).
Составить словарь с ключами - точками (например, Point3D(3, 2, 1)) и значениями - именами точек (G, H и т.д.).

 
Найти и вывести на экран


расстояние от точек R и J до плоскости GHK,

 
уравнение плоскости GHK,

 
параметрические уравнения прямой RJ.


Определить и вывести на экран, какие 4 из заданных точек лежат в одной плоскости (пользуясь словарем точек, вывести имена точек).


Определить и вывести на экран, какие 3 из заданных точек лежат на одной прямой (пользуясь словарем точек, вывести имена точек).

Вариант N 119

Даны точки: H (0, 0, 0), K (-6, -6, 4), P (-3, -3, 2), S (0, 0, 1), T (1, 5, -3).
Составить словарь с ключами - точками (например, Point3D(3, 2, 1)) и значениями - именами точек (H, P и т.д.).

 
Найти и вывести на экран


расстояние от точек T и K до плоскости HPS,

 
уравнение плоскости HPS,

 
параметрические уравнения прямой TK.


Определить и вывести на экран, какие 4 из заданных точек лежат в одной плоскости (пользуясь словарем точек, вывести имена точек).


Определить и вывести на экран, какие 3 из заданных точек лежат на одной прямой (пользуясь словарем точек, вывести имена точек).

Вариант N 120

Даны точки: F (-4, -3, 2), H (0, 3, -4), J (4, 5, 3), Q (-1, -4, -1), R (-3, -18, 5).
Составить словарь с ключами - точками (например, Point3D(3, 2, 1)) и значениями - именами точек (F, H и т.д.).

 
Найти и вывести на экран


расстояние от точек J и R до плоскости FHQ,

 
уравнение плоскости FHQ,

 
параметрические уравнения прямой JR.


Определить и вывести на экран, какие 4 из заданных точек лежат в одной плоскости (пользуясь словарем точек, вывести имена точек).


Определить и вывести на экран, какие 3 из заданных точек лежат на одной прямой (пользуясь словарем точек, вывести имена точек).

Вариант N 121

Даны точки: F (2, 4, 0), G (4, 1, 3), L (2, 5, 1), N (14, -19, 13), Q (3, 5, 5).
Составить словарь с ключами - точками (например, Point3D(3, 2, 1)) и значениями - именами точек (G, L и т.д.).

 
Найти и вывести на экран


расстояние от точек F и N до плоскости GLQ,

 
уравнение плоскости GLQ,

 
параметрические уравнения прямой FN.


Определить и вывести на экран, какие 4 из заданных точек лежат в одной плоскости (пользуясь словарем точек, вывести имена точек).


Определить и вывести на экран, какие 3 из заданных точек лежат на одной прямой (пользуясь словарем точек, вывести имена точек).

Вариант N 122

Даны точки: F (-3, -2, -1), G (5, -2, 3), H (5, -1, -2), N (17, -11, -14), T (-1, 4, 4).
Составить словарь с ключами - точками (например, Point3D(3, 2, 1)) и значениями - именами точек (G, H и т.д.).

 
Найти и вывести на экран


расстояние от точек F и N до плоскости GHT,

 
уравнение плоскости GHT,

 
параметрические уравнения прямой FN.


Определить и вывести на экран, какие 4 из заданных точек лежат в одной плоскости (пользуясь словарем точек, вывести имена точек).


Определить и вывести на экран, какие 3 из заданных точек лежат на одной прямой (пользуясь словарем точек, вывести имена точек).

Вариант N 123

Даны точки: K (1, 5, 0), L (5, 2, 3), Q (12, -4, 3), S (-3, -4, 0), T (2, -4, 1).
Составить словарь с ключами - точками (например, Point3D(3, 2, 1)) и значениями - именами точек (L, S и т.д.).

 
Найти и вывести на экран


расстояние от точек K и Q до плоскости LST,

 
уравнение плоскости LST,

 
параметрические уравнения прямой KQ.


Определить и вывести на экран, какие 4 из заданных точек лежат в одной плоскости (пользуясь словарем точек, вывести имена точек).


Определить и вывести на экран, какие 3 из заданных точек лежат на одной прямой (пользуясь словарем точек, вывести имена точек).

Вариант N 124

Даны точки: F (0, -1, 2), L (2, 3, 5), R (-1, 6, -16), S (1, 4, -2), T (1, 1, -4).
Составить словарь с ключами - точками (например, Point3D(3, 2, 1)) и значениями - именами точек (L, S и т.д.).

 
Найти и вывести на экран


расстояние от точек F и R до плоскости LST,

 
уравнение плоскости LST,

 
параметрические уравнения прямой FR.


Определить и вывести на экран, какие 4 из заданных точек лежат в одной плоскости (пользуясь словарем точек, вывести имена точек).


Определить и вывести на экран, какие 3 из заданных точек лежат на одной прямой (пользуясь словарем точек, вывести имена точек).

Вариант N 125

Даны точки: H (2, 0, 2), M (1, 5, -1), P (4, 4, 0), R (-6, -16, 10), T (1, 4, -3).
Составить словарь с ключами - точками (например, Point3D(3, 2, 1)) и значениями - именами точек (H, P и т.д.).

 
Найти и вывести на экран


расстояние от точек M и R до плоскости HPT,

 
уравнение плоскости HPT,

 
параметрические уравнения прямой MR.


Определить и вывести на экран, какие 4 из заданных точек лежат в одной плоскости (пользуясь словарем точек, вывести имена точек).


Определить и вывести на экран, какие 3 из заданных точек лежат на одной прямой (пользуясь словарем точек, вывести имена точек).

Вариант N 126

Даны точки: F (1, 4, 0), J (-5, -1, 15), K (4, -1, 0), R (2, -2, 5), S (1, -1, 5).
Составить словарь с ключами - точками (например, Point3D(3, 2, 1)) и значениями - именами точек (F, K и т.д.).

 
Найти и вывести на экран


расстояние от точек R и J до плоскости FKS,

 
уравнение плоскости FKS,

 
параметрические уравнения прямой RJ.


Определить и вывести на экран, какие 4 из заданных точек лежат в одной плоскости (пользуясь словарем точек, вывести имена точек).


Определить и вывести на экран, какие 3 из заданных точек лежат на одной прямой (пользуясь словарем точек, вывести имена точек).

Вариант N 127

Даны точки: F (-4, -1, 2), G (-3, 3, -4), K (-1, 5, 3), P (-1, -4, 2), R (-11, 31, -28).
Составить словарь с ключами - точками (например, Point3D(3, 2, 1)) и значениями - именами точек (G, K и т.д.).

 
Найти и вывести на экран


расстояние от точек F и R до плоскости GKP,

 
уравнение плоскости GKP,

 
параметрические уравнения прямой FR.


Определить и вывести на экран, какие 4 из заданных точек лежат в одной плоскости (пользуясь словарем точек, вывести имена точек).


Определить и вывести на экран, какие 3 из заданных точек лежат на одной прямой (пользуясь словарем точек, вывести имена точек).

Вариант N 128

Даны точки: F (-1, 1, 2), H (-2, -2, -3), J (-2, 6, 13), P (-2, 2, 5), R (4, -1, -3).
Составить словарь с ключами - точками (например, Point3D(3, 2, 1)) и значениями - именами точек (F, H и т.д.).

 
Найти и вывести на экран


расстояние от точек R и J до плоскости FHP,

 
уравнение плоскости FHP,

 
параметрические уравнения прямой RJ.


Определить и вывести на экран, какие 4 из заданных точек лежат в одной плоскости (пользуясь словарем точек, вывести имена точек).


Определить и вывести на экран, какие 3 из заданных точек лежат на одной прямой (пользуясь словарем точек, вывести имена точек).

Вариант N 129

Даны точки: G (-1, -2, 4), J (29, 10, 4), N (-4, -4, -3), R (4, -1, -4), S (4, 0, 4).
Составить словарь с ключами - точками (например, Point3D(3, 2, 1)) и значениями - именами точек (G, R и т.д.).

 
Найти и вывести на экран


расстояние от точек N и J до плоскости GRS,

 
уравнение плоскости GRS,

 
параметрические уравнения прямой NJ.


Определить и вывести на экран, какие 4 из заданных точек лежат в одной плоскости (пользуясь словарем точек, вывести имена точек).


Определить и вывести на экран, какие 3 из заданных точек лежат на одной прямой (пользуясь словарем точек, вывести имена точек).

Вариант N 130

Даны точки: G (1, 4, -2), H (-2, 0, 1), N (-7, 10, 6), P (-3, 2, 2), T (-1, -2, -2).
Составить словарь с ключами - точками (например, Point3D(3, 2, 1)) и значениями - именами точек (H, P и т.д.).

 
Найти и вывести на экран


расстояние от точек G и N до плоскости HPT,

 
уравнение плоскости HPT,

 
параметрические уравнения прямой GN.


Определить и вывести на экран, какие 4 из заданных точек лежат в одной плоскости (пользуясь словарем точек, вывести имена точек).


Определить и вывести на экран, какие 3 из заданных точек лежат на одной прямой (пользуясь словарем точек, вывести имена точек).

Вариант N 131

Даны точки: H (1, 5, -1), K (-3, 0, 3), M (2, 4, -4), N (-15, 25, 3), P (5, 0, -2).
Составить словарь с ключами - точками (например, Point3D(3, 2, 1)) и значениями - именами точек (H, M и т.д.).

 
Найти и вывести на экран


расстояние от точек K и N до плоскости HMP,

 
уравнение плоскости HMP,

 
параметрические уравнения прямой KN.


Определить и вывести на экран, какие 4 из заданных точек лежат в одной плоскости (пользуясь словарем точек, вывести имена точек).


Определить и вывести на экран, какие 3 из заданных точек лежат на одной прямой (пользуясь словарем точек, вывести имена точек).

Вариант N 132

Даны точки: G (-2, -2, -1), H (5, 2, -1), P (-4, 0, 5), R (23, 6, -13), S (5, 3, -2).
Составить словарь с ключами - точками (например, Point3D(3, 2, 1)) и значениями - именами точек (H, P и т.д.).

 
Найти и вывести на экран


расстояние от точек G и R до плоскости HPS,

 
уравнение плоскости HPS,

 
параметрические уравнения прямой GR.


Определить и вывести на экран, какие 4 из заданных точек лежат в одной плоскости (пользуясь словарем точек, вывести имена точек).


Определить и вывести на экран, какие 3 из заданных точек лежат на одной прямой (пользуясь словарем точек, вывести имена точек).

Вариант N 133

Даны точки: F (-2, -3, 4), L (4, 2, -2), N (32, 6, -6), R (4, -3, -3), S (-3, 1, -1).
Составить словарь с ключами - точками (например, Point3D(3, 2, 1)) и значениями - именами точек (F, L и т.д.).

 
Найти и вывести на экран


расстояние от точек R и N до плоскости FLS,

 
уравнение плоскости FLS,

 
параметрические уравнения прямой RN.


Определить и вывести на экран, какие 4 из заданных точек лежат в одной плоскости (пользуясь словарем точек, вывести имена точек).


Определить и вывести на экран, какие 3 из заданных точек лежат на одной прямой (пользуясь словарем точек, вывести имена точек).

Вариант N 134

Даны точки: F (0, 0, 5), H (2, -2, 1), L (4, 2, 1), N (-18, -6, -11), T (-3, -3, -2).
Составить словарь с ключами - точками (например, Point3D(3, 2, 1)) и значениями - именами точек (H, L и т.д.).

 
Найти и вывести на экран


расстояние от точек F и N до плоскости HLT,

 
уравнение плоскости HLT,

 
параметрические уравнения прямой FN.


Определить и вывести на экран, какие 4 из заданных точек лежат в одной плоскости (пользуясь словарем точек, вывести имена точек).


Определить и вывести на экран, какие 3 из заданных точек лежат на одной прямой (пользуясь словарем точек, вывести имена точек).

Вариант N 135

Даны точки: H (0, 1, -1), K (0, 2, 2), L (1, -3, 4), N (10, -19, 14), P (2, -3, 2).
Составить словарь с ключами - точками (например, Point3D(3, 2, 1)) и значениями - именами точек (H, L и т.д.).

 
Найти и вывести на экран


расстояние от точек K и N до плоскости HLP,

 
уравнение плоскости HLP,

 
параметрические уравнения прямой KN.


Определить и вывести на экран, какие 4 из заданных точек лежат в одной плоскости (пользуясь словарем точек, вывести имена точек).


Определить и вывести на экран, какие 3 из заданных точек лежат на одной прямой (пользуясь словарем точек, вывести имена точек).

Вариант N 136

Даны точки: H (3, 1, 3), K (3, -4, 1), L (-4, 0, 4), Q (-7, 6, -22), S (1, 2, -2).
Составить словарь с ключами - точками (например, Point3D(3, 2, 1)) и значениями - именами точек (H, L и т.д.).

 
Найти и вывести на экран


расстояние от точек K и Q до плоскости HLS,

 
уравнение плоскости HLS,

 
параметрические уравнения прямой KQ.


Определить и вывести на экран, какие 4 из заданных точек лежат в одной плоскости (пользуясь словарем точек, вывести имена точек).


Определить и вывести на экран, какие 3 из заданных точек лежат на одной прямой (пользуясь словарем точек, вывести имена точек).

Вариант N 137

Даны точки: J (-5, -8, 6), K (-3, 0, 3), M (1, -2, 0), P (3, 0, -2), Q (2, -2, 1).
Составить словарь с ключами - точками (например, Point3D(3, 2, 1)) и значениями - именами точек (K, M и т.д.).

 
Найти и вывести на экран


расстояние от точек Q и J до плоскости KMP,

 
уравнение плоскости KMP,

 
параметрические уравнения прямой QJ.


Определить и вывести на экран, какие 4 из заданных точек лежат в одной плоскости (пользуясь словарем точек, вывести имена точек).


Определить и вывести на экран, какие 3 из заданных точек лежат на одной прямой (пользуясь словарем точек, вывести имена точек).

Вариант N 138

Даны точки: G (1, -3, 5), J (-7, -11, 19), K (1, 4, 4), N (-2, -1, 0), P (5, 1, -2).
Составить словарь с ключами - точками (например, Point3D(3, 2, 1)) и значениями - именами точек (G, K и т.д.).

 
Найти и вывести на экран


расстояние от точек N и J до плоскости GKP,

 
уравнение плоскости GKP,

 
параметрические уравнения прямой NJ.


Определить и вывести на экран, какие 4 из заданных точек лежат в одной плоскости (пользуясь словарем точек, вывести имена точек).


Определить и вывести на экран, какие 3 из заданных точек лежат на одной прямой (пользуясь словарем точек, вывести имена точек).

Вариант N 139

Даны точки: F (-4, 1, 1), H (3, -3, 3), M (5, -1, -1), N (18, -18, 21), S (-2, 2, -3).
Составить словарь с ключами - точками (например, Point3D(3, 2, 1)) и значениями - именами точек (H, M и т.д.).

 
Найти и вывести на экран


расстояние от точек F и N до плоскости HMS,

 
уравнение плоскости HMS,

 
параметрические уравнения прямой FN.


Определить и вывести на экран, какие 4 из заданных точек лежат в одной плоскости (пользуясь словарем точек, вывести имена точек).


Определить и вывести на экран, какие 3 из заданных точек лежат на одной прямой (пользуясь словарем точек, вывести имена точек).

Вариант N 140

Даны точки: J (16, 12, -4), L (1, 3, -4), N (3, 4, 2), P (-4, 0, -4), R (1, 0, 4).
Составить словарь с ключами - точками (например, Point3D(3, 2, 1)) и значениями - именами точек (L, P и т.д.).

 
Найти и вывести на экран


расстояние от точек N и J до плоскости LPR,

 
уравнение плоскости LPR,

 
параметрические уравнения прямой NJ.


Определить и вывести на экран, какие 4 из заданных точек лежат в одной плоскости (пользуясь словарем точек, вывести имена точек).


Определить и вывести на экран, какие 3 из заданных точек лежат на одной прямой (пользуясь словарем точек, вывести имена точек).

Вариант N 141

Даны точки: K (0, 0, -3), N (14, 26, 2), P (4, 1, 2), Q (5, -1, 2), S (2, -4, 2).
Составить словарь с ключами - точками (например, Point3D(3, 2, 1)) и значениями - именами точек (K, P и т.д.).

 
Найти и вывести на экран


расстояние от точек Q и N до плоскости KPS,

 
уравнение плоскости KPS,

 
параметрические уравнения прямой QN.


Определить и вывести на экран, какие 4 из заданных точек лежат в одной плоскости (пользуясь словарем точек, вывести имена точек).


Определить и вывести на экран, какие 3 из заданных точек лежат на одной прямой (пользуясь словарем точек, вывести имена точек).

Вариант N 142

Даны точки: H (5, -4, 2), K (5, 3, 1), L (2, -2, 5), R (-13, 14, 6), S (-4, 5, 4).
Составить словарь с ключами - точками (например, Point3D(3, 2, 1)) и значениями - именами точек (H, L и т.д.).

 
Найти и вывести на экран


расстояние от точек K и R до плоскости HLS,

 
уравнение плоскости HLS,

 
параметрические уравнения прямой KR.


Определить и вывести на экран, какие 4 из заданных точек лежат в одной плоскости (пользуясь словарем точек, вывести имена точек).


Определить и вывести на экран, какие 3 из заданных точек лежат на одной прямой (пользуясь словарем точек, вывести имена точек).

Вариант N 143

Даны точки: F (5, 3, 2), L (-4, 3, -3), M (5, -3, -2), N (-4, 3, 6), S (-4, 3, 0).
Составить словарь с ключами - точками (например, Point3D(3, 2, 1)) и значениями - именами точек (L, M и т.д.).

 
Найти и вывести на экран


расстояние от точек F и N до плоскости LMS,

 
уравнение плоскости LMS,

 
параметрические уравнения прямой FN.


Определить и вывести на экран, какие 4 из заданных точек лежат в одной плоскости (пользуясь словарем точек, вывести имена точек).


Определить и вывести на экран, какие 3 из заданных точек лежат на одной прямой (пользуясь словарем точек, вывести имена точек).

Вариант N 144

Даны точки: H (4, -4, 5), J (-28, 32, -31), M (-1, 5, -1), P (-4, 5, -4), Q (5, 0, -4).
Составить словарь с ключами - точками (например, Point3D(3, 2, 1)) и значениями - именами точек (H, M и т.д.).

 
Найти и вывести на экран


расстояние от точек Q и J до плоскости HMP,

 
уравнение плоскости HMP,

 
параметрические уравнения прямой QJ.


Определить и вывести на экран, какие 4 из заданных точек лежат в одной плоскости (пользуясь словарем точек, вывести имена точек).


Определить и вывести на экран, какие 3 из заданных точек лежат на одной прямой (пользуясь словарем точек, вывести имена точек).

Вариант N 145

Даны точки: F (-2, -3, 3), G (-1, 1, 2), H (2, -4, -2), J (14, -16, 2), P (-1, -1, -3).
Составить словарь с ключами - точками (например, Point3D(3, 2, 1)) и значениями - именами точек (G, H и т.д.).

 
Найти и вывести на экран


расстояние от точек F и J до плоскости GHP,

 
уравнение плоскости GHP,

 
параметрические уравнения прямой FJ.


Определить и вывести на экран, какие 4 из заданных точек лежат в одной плоскости (пользуясь словарем точек, вывести имена точек).


Определить и вывести на экран, какие 3 из заданных точек лежат на одной прямой (пользуясь словарем точек, вывести имена точек).

Вариант N 146

Даны точки: F (1, -13, -20), G (0, 0, 5), K (1, 0, 2), M (3, -1, -4), P (4, 5, 4).
Составить словарь с ключами - точками (например, Point3D(3, 2, 1)) и значениями - именами точек (G, M и т.д.).

 
Найти и вывести на экран


расстояние от точек K и F до плоскости GMP,

 
уравнение плоскости GMP,

 
параметрические уравнения прямой KF.


Определить и вывести на экран, какие 4 из заданных точек лежат в одной плоскости (пользуясь словарем точек, вывести имена точек).


Определить и вывести на экран, какие 3 из заданных точек лежат на одной прямой (пользуясь словарем точек, вывести имена точек).

Вариант N 147

Даны точки: H (-3, 5, 1), J (-18, 50, -19), L (5, -4, 4), R (3, 2, 1), T (0, -4, 5).
Составить словарь с ключами - точками (например, Point3D(3, 2, 1)) и значениями - именами точек (H, L и т.д.).

 
Найти и вывести на экран


расстояние от точек R и J до плоскости HLT,

 
уравнение плоскости HLT,

 
параметрические уравнения прямой RJ.


Определить и вывести на экран, какие 4 из заданных точек лежат в одной плоскости (пользуясь словарем точек, вывести имена точек).


Определить и вывести на экран, какие 3 из заданных точек лежат на одной прямой (пользуясь словарем точек, вывести имена точек).

Вариант N 148

Даны точки: F (-4, 4, 1), G (-2, -1, -3), H (-3, 5, 1), M (5, 4, 2), R (21, 2, 4).
Составить словарь с ключами - точками (например, Point3D(3, 2, 1)) и значениями - именами точек (G, H и т.д.).

 
Найти и вывести на экран


расстояние от точек F и R до плоскости GHM,

 
уравнение плоскости GHM,

 
параметрические уравнения прямой FR.


Определить и вывести на экран, какие 4 из заданных точек лежат в одной плоскости (пользуясь словарем точек, вывести имена точек).


Определить и вывести на экран, какие 3 из заданных точек лежат на одной прямой (пользуясь словарем точек, вывести имена точек).

Вариант N 149

Даны точки: G (-3, 2, -4), J (-3, 6, 20), M (-2, 3, -3), P (-3, 3, 2), Q (1, -1, 4).
Составить словарь с ключами - точками (например, Point3D(3, 2, 1)) и значениями - именами точек (G, M и т.д.).

 
Найти и вывести на экран


расстояние от точек Q и J до плоскости GMP,

 
уравнение плоскости GMP,

 
параметрические уравнения прямой QJ.


Определить и вывести на экран, какие 4 из заданных точек лежат в одной плоскости (пользуясь словарем точек, вывести имена точек).


Определить и вывести на экран, какие 3 из заданных точек лежат на одной прямой (пользуясь словарем точек, вывести имена точек).

Вариант N 150

Даны точки: J (21, -25, -19), K (5, 0, -1), L (5, -1, -3), P (1, 5, 1), Q (2, 5, 2).
Составить словарь с ключами - точками (например, Point3D(3, 2, 1)) и значениями - именами точек (K, L и т.д.).

 
Найти и вывести на экран


расстояние от точек Q и J до плоскости KLP,

 
уравнение плоскости KLP,

 
параметрические уравнения прямой QJ.


Определить и вывести на экран, какие 4 из заданных точек лежат в одной плоскости (пользуясь словарем точек, вывести имена точек).


Определить и вывести на экран, какие 3 из заданных точек лежат на одной прямой (пользуясь словарем точек, вывести имена точек).

\end{document}