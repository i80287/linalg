 \documentclass[11pt]{report}

\usepackage[T2A]{fontenc}

\usepackage[utf8]{inputenc}

\usepackage[russian]{babel}

\usepackage{amsmath,amssymb}

\usepackage{graphicx}

\graphicspath{ {d:/HSE/OR/CW/CW5pict/} }

\begin{document}

\pagestyle{empty}

{\bf Индивидуальное задание.}

Вариант N 1

Эллипс с центром $Point2D\left(-3, 3\right)$, вертикальной полуосью $2$, эксцентриситетом $5 / 9$.

    Изобразить на графике эллипс, а также эллипс, повернутый на угол $\alpha = $$\pi / 3$ по часовой стрелке.

Вариант N 2

Эллипс с центром $Point2D\left(-3, 5\right)$, горизонтальной полуосью $27 \sqrt{5} / 5$, эксцентриситетом $2 / 3$.

    Изобразить на графике эллипс, а также эллипс, повернутый на угол $\alpha = $$\pi / 6$ против часовой стрелки.

Вариант N 3

Эллипс с центром $Point2D\left(1, 3\right)$, вертикальной полуосью $3$, эксцентриситетом $1 / 2$.

    Изобразить на графике эллипс, а также эллипс, повернутый на угол $\alpha = $$\pi / 4$ по часовой стрелке.

Вариант N 4

Эллипс с центром $Point2D\left(1, 4\right)$, горизонтальной полуосью $4 \sqrt{7}$, эксцентриситетом $3 / 4$.

    Изобразить на графике эллипс, а также эллипс, повернутый на угол $\alpha = $$\pi / 3$ против часовой стрелки.

Вариант N 5

Эллипс с центром $Point2D\left(4, -3\right)$, вертикальной полуосью $8$, эксцентриситетом $8 / 11$.

    Изобразить на графике эллипс, а также эллипс, повернутый на угол $\alpha = $$\pi / 3$ по часовой стрелке.

Вариант N 6

Эллипс с центром $Point2D\left(1, 4\right)$, горизонтальной полуосью $24 \sqrt{15} / 5$, эксцентриситетом $7 / 8$.

    Изобразить на графике эллипс, а также эллипс, повернутый на угол $\alpha = $$\pi / 3$ против часовой стрелки.

Вариант N 7

Эллипс с центром $Point2D\left(-2, -1\right)$, вертикальной полуосью $2$, эксцентриситетом $9 / 10$.

    Изобразить на графике эллипс, а также эллипс, повернутый на угол $\alpha = $$\pi / 3$ по часовой стрелке.

Вариант N 8

Эллипс с центром $Point2D\left(-4, -4\right)$, горизонтальной полуосью $119 / 15$, эксцентриситетом $8 / 17$.

    Изобразить на графике эллипс, а также эллипс, повернутый на угол $\alpha = $$\pi / 3$ против часовой стрелки.

Вариант N 9

Эллипс с центром $Point2D\left(5, -4\right)$, вертикальной полуосью $9$, эксцентриситетом $9 / 16$.

    Изобразить на графике эллипс, а также эллипс, повернутый на угол $\alpha = $$\pi / 3$ по часовой стрелке.

Вариант N 10

Эллипс с центром $Point2D\left(-3, -3\right)$, горизонтальной полуосью $32 \sqrt{7} / 7$, эксцентриситетом $3 / 4$.

    Изобразить на графике эллипс, а также эллипс, повернутый на угол $\alpha = $$\pi / 6$ против часовой стрелки.

Вариант N 11

Эллипс с центром $Point2D\left(-2, 1\right)$, вертикальной полуосью $8$, эксцентриситетом $6 / 13$.

    Изобразить на графике эллипс, а также эллипс, повернутый на угол $\alpha = $$\pi / 3$ по часовой стрелке.

Вариант N 12

Эллипс с центром $Point2D\left(3, -1\right)$, горизонтальной полуосью $3 \sqrt{5}$, эксцентриситетом $2 / 3$.

    Изобразить на графике эллипс, а также эллипс, повернутый на угол $\alpha = $$\pi / 6$ против часовой стрелки.

Вариант N 13

Эллипс с центром $Point2D\left(0, -3\right)$, вертикальной полуосью $9$, эксцентриситетом $5 / 12$.

    Изобразить на графике эллипс, а также эллипс, повернутый на угол $\alpha = $$\pi / 3$ по часовой стрелке.

Вариант N 14

Эллипс с центром $Point2D\left(-4, 3\right)$, горизонтальной полуосью $63 \sqrt{10} / 20$, эксцентриситетом $3 / 7$.

    Изобразить на графике эллипс, а также эллипс, повернутый на угол $\alpha = $$\pi / 3$ против часовой стрелки.

Вариант N 15

Эллипс с центром $Point2D\left(0, 3\right)$, вертикальной полуосью $9$, эксцентриситетом $1 / 4$.

    Изобразить на графике эллипс, а также эллипс, повернутый на угол $\alpha = $$\pi / 4$ по часовой стрелке.

Вариант N 16

Эллипс с центром $Point2D\left(-4, 3\right)$, горизонтальной полуосью $20 \sqrt{7} / 7$, эксцентриситетом $3 / 4$.

    Изобразить на графике эллипс, а также эллипс, повернутый на угол $\alpha = $$\pi / 4$ против часовой стрелки.

Вариант N 17

Эллипс с центром $Point2D\left(-1, 4\right)$, вертикальной полуосью $9$, эксцентриситетом $4 / 11$.

    Изобразить на графике эллипс, а также эллипс, повернутый на угол $\alpha = $$\pi / 4$ по часовой стрелке.

Вариант N 18

Эллипс с центром $Point2D\left(-1, 4\right)$, горизонтальной полуосью $7 \sqrt{10} / 4$, эксцентриситетом $3 / 7$.

    Изобразить на графике эллипс, а также эллипс, повернутый на угол $\alpha = $$\pi / 4$ против часовой стрелки.

Вариант N 19

Эллипс с центром $Point2D\left(-3, 1\right)$, вертикальной полуосью $8$, эксцентриситетом $8 / 9$.

    Изобразить на графике эллипс, а также эллипс, повернутый на угол $\alpha = $$\pi / 3$ по часовой стрелке.

Вариант N 20

Эллипс с центром $Point2D\left(-1, -3\right)$, горизонтальной полуосью $26 \sqrt{17} / 51$, эксцентриситетом $4 / 13$.

    Изобразить на графике эллипс, а также эллипс, повернутый на угол $\alpha = $$\pi / 4$ против часовой стрелки.

Вариант N 21

Эллипс с центром $Point2D\left(2, -4\right)$, вертикальной полуосью $6$, эксцентриситетом $8 / 13$.

    Изобразить на графике эллипс, а также эллипс, повернутый на угол $\alpha = $$\pi / 4$ по часовой стрелке.

Вариант N 22

Эллипс с центром $Point2D\left(-2, 1\right)$, горизонтальной полуосью $35 \sqrt{6} / 12$, эксцентриситетом $1 / 5$.

    Изобразить на графике эллипс, а также эллипс, повернутый на угол $\alpha = $$\pi / 6$ против часовой стрелки.

Вариант N 23

Эллипс с центром $Point2D\left(3, 1\right)$, вертикальной полуосью $5$, эксцентриситетом $2 / 7$.

    Изобразить на графике эллипс, а также эллипс, повернутый на угол $\alpha = $$\pi / 3$ по часовой стрелке.

Вариант N 24

Эллипс с центром $Point2D\left(-1, 5\right)$, горизонтальной полуосью $56 \sqrt{115} / 115$, эксцентриситетом $9 / 14$.

    Изобразить на графике эллипс, а также эллипс, повернутый на угол $\alpha = $$\pi / 4$ против часовой стрелки.

Вариант N 25

Эллипс с центром $Point2D\left(5, 5\right)$, вертикальной полуосью $8$, эксцентриситетом $9 / 10$.

    Изобразить на графике эллипс, а также эллипс, повернутый на угол $\alpha = $$\pi / 4$ по часовой стрелке.

Вариант N 26

Эллипс с центром $Point2D\left(4, 5\right)$, горизонтальной полуосью $10 \sqrt{3} / 3$, эксцентриситетом $1 / 2$.

    Изобразить на графике эллипс, а также эллипс, повернутый на угол $\alpha = $$\pi / 6$ против часовой стрелки.

Вариант N 27

Эллипс с центром $Point2D\left(2, 1\right)$, вертикальной полуосью $3$, эксцентриситетом $5 / 7$.

    Изобразить на графике эллипс, а также эллипс, повернутый на угол $\alpha = $$\pi / 4$ по часовой стрелке.

Вариант N 28

Эллипс с центром $Point2D\left(5, 4\right)$, горизонтальной полуосью $13 \sqrt{30} / 10$, эксцентриситетом $7 / 13$.

    Изобразить на графике эллипс, а также эллипс, повернутый на угол $\alpha = $$\pi / 3$ против часовой стрелки.

Вариант N 29

Эллипс с центром $Point2D\left(-1, 3\right)$, вертикальной полуосью $4$, эксцентриситетом $3 / 5$.

    Изобразить на графике эллипс, а также эллипс, повернутый на угол $\alpha = $$\pi / 6$ по часовой стрелке.

Вариант N 30

Эллипс с центром $Point2D\left(-1, -1\right)$, горизонтальной полуосью $32 \sqrt{7} / 7$, эксцентриситетом $3 / 4$.

    Изобразить на графике эллипс, а также эллипс, повернутый на угол $\alpha = $$\pi / 3$ против часовой стрелки.

Вариант N 31

Эллипс с центром $Point2D\left(4, 4\right)$, вертикальной полуосью $4$, эксцентриситетом $2 / 3$.

    Изобразить на графике эллипс, а также эллипс, повернутый на угол $\alpha = $$\pi / 6$ по часовой стрелке.

Вариант N 32

Эллипс с центром $Point2D\left(5, 0\right)$, горизонтальной полуосью $24 \sqrt{55} / 55$, эксцентриситетом $3 / 8$.

    Изобразить на графике эллипс, а также эллипс, повернутый на угол $\alpha = $$\pi / 4$ против часовой стрелки.

Вариант N 33

Эллипс с центром $Point2D\left(5, -1\right)$, вертикальной полуосью $2$, эксцентриситетом $5 / 6$.

    Изобразить на графике эллипс, а также эллипс, повернутый на угол $\alpha = $$\pi / 3$ по часовой стрелке.

Вариант N 34

Эллипс с центром $Point2D\left(-1, 1\right)$, горизонтальной полуосью $13 \sqrt{30} / 20$, эксцентриситетом $7 / 13$.

    Изобразить на графике эллипс, а также эллипс, повернутый на угол $\alpha = $$\pi / 4$ против часовой стрелки.

Вариант N 35

Эллипс с центром $Point2D\left(2, 4\right)$, вертикальной полуосью $8$, эксцентриситетом $2 / 9$.

    Изобразить на графике эллипс, а также эллипс, повернутый на угол $\alpha = $$\pi / 3$ по часовой стрелке.

Вариант N 36

Эллипс с центром $Point2D\left(4, -2\right)$, горизонтальной полуосью $13 / 3$, эксцентриситетом $5 / 13$.

    Изобразить на графике эллипс, а также эллипс, повернутый на угол $\alpha = $$\pi / 3$ против часовой стрелки.

Вариант N 37

Эллипс с центром $Point2D\left(-3, 2\right)$, вертикальной полуосью $6$, эксцентриситетом $6 / 11$.

    Изобразить на графике эллипс, а также эллипс, повернутый на угол $\alpha = $$\pi / 6$ по часовой стрелке.

Вариант N 38

Эллипс с центром $Point2D\left(-2, 1\right)$, горизонтальной полуосью $18 \sqrt{5} / 5$, эксцентриситетом $2 / 3$.

    Изобразить на графике эллипс, а также эллипс, повернутый на угол $\alpha = $$\pi / 4$ против часовой стрелки.

Вариант N 39

Эллипс с центром $Point2D\left(2, 0\right)$, вертикальной полуосью $4$, эксцентриситетом $2 / 3$.

    Изобразить на графике эллипс, а также эллипс, повернутый на угол $\alpha = $$\pi / 3$ по часовой стрелке.

Вариант N 40

Эллипс с центром $Point2D\left(-3, 1\right)$, горизонтальной полуосью $104 \sqrt{105} / 105$, эксцентриситетом $8 / 13$.

    Изобразить на графике эллипс, а также эллипс, повернутый на угол $\alpha = $$\pi / 4$ против часовой стрелки.

Вариант N 41

Эллипс с центром $Point2D\left(1, -4\right)$, вертикальной полуосью $9$, эксцентриситетом $3 / 7$.

    Изобразить на графике эллипс, а также эллипс, повернутый на угол $\alpha = $$\pi / 4$ по часовой стрелке.

Вариант N 42

Эллипс с центром $Point2D\left(-4, 4\right)$, горизонтальной полуосью $51 \sqrt{13} / 52$, эксцентриситетом $9 / 17$.

    Изобразить на графике эллипс, а также эллипс, повернутый на угол $\alpha = $$\pi / 3$ против часовой стрелки.

Вариант N 43

Эллипс с центром $Point2D\left(-1, 4\right)$, вертикальной полуосью $7$, эксцентриситетом $2 / 3$.

    Изобразить на графике эллипс, а также эллипс, повернутый на угол $\alpha = $$\pi / 3$ по часовой стрелке.

Вариант N 44

Эллипс с центром $Point2D\left(-1, 1\right)$, горизонтальной полуосью $120 \sqrt{161} / 161$, эксцентриситетом $8 / 15$.

    Изобразить на графике эллипс, а также эллипс, повернутый на угол $\alpha = $$\pi / 6$ против часовой стрелки.

Вариант N 45

Эллипс с центром $Point2D\left(0, 1\right)$, вертикальной полуосью $8$, эксцентриситетом $1 / 3$.

    Изобразить на графике эллипс, а также эллипс, повернутый на угол $\alpha = $$\pi / 3$ по часовой стрелке.

Вариант N 46

Эллипс с центром $Point2D\left(-2, 2\right)$, горизонтальной полуосью $88 \sqrt{13} / 39$, эксцентриситетом $2 / 11$.

    Изобразить на графике эллипс, а также эллипс, повернутый на угол $\alpha = $$\pi / 6$ против часовой стрелки.

Вариант N 47

Эллипс с центром $Point2D\left(-4, 0\right)$, вертикальной полуосью $6$, эксцентриситетом $7 / 10$.

    Изобразить на графике эллипс, а также эллипс, повернутый на угол $\alpha = $$\pi / 6$ по часовой стрелке.

Вариант N 48

Эллипс с центром $Point2D\left(-4, -4\right)$, горизонтальной полуосью $16 \sqrt{7} / 7$, эксцентриситетом $3 / 4$.

    Изобразить на графике эллипс, а также эллипс, повернутый на угол $\alpha = $$\pi / 6$ против часовой стрелки.

Вариант N 49

Эллипс с центром $Point2D\left(2, 3\right)$, вертикальной полуосью $9$, эксцентриситетом $8 / 15$.

    Изобразить на графике эллипс, а также эллипс, повернутый на угол $\alpha = $$\pi / 3$ по часовой стрелке.

Вариант N 50

Эллипс с центром $Point2D\left(3, -1\right)$, горизонтальной полуосью $33 \sqrt{57} / 19$, эксцентриситетом $8 / 11$.

    Изобразить на графике эллипс, а также эллипс, повернутый на угол $\alpha = $$\pi / 6$ против часовой стрелки.

Вариант N 51

Эллипс с центром $Point2D\left(3, -4\right)$, вертикальной полуосью $6$, эксцентриситетом $1 / 3$.

    Изобразить на графике эллипс, а также эллипс, повернутый на угол $\alpha = $$\pi / 6$ по часовой стрелке.

Вариант N 52

Эллипс с центром $Point2D\left(1, 5\right)$, горизонтальной полуосью $14 \sqrt{33} / 11$, эксцентриситетом $4 / 7$.

    Изобразить на графике эллипс, а также эллипс, повернутый на угол $\alpha = $$\pi / 4$ против часовой стрелки.

Вариант N 53

Эллипс с центром $Point2D\left(3, 1\right)$, вертикальной полуосью $2$, эксцентриситетом $4 / 5$.

    Изобразить на графике эллипс, а также эллипс, повернутый на угол $\alpha = $$\pi / 4$ по часовой стрелке.

Вариант N 54

Эллипс с центром $Point2D\left(1, -1\right)$, горизонтальной полуосью $10 \sqrt{21} / 7$, эксцентриситетом $2 / 5$.

    Изобразить на графике эллипс, а также эллипс, повернутый на угол $\alpha = $$\pi / 4$ против часовой стрелки.

Вариант N 55

Эллипс с центром $Point2D\left(3, 1\right)$, вертикальной полуосью $8$, эксцентриситетом $2 / 9$.

    Изобразить на графике эллипс, а также эллипс, повернутый на угол $\alpha = $$\pi / 3$ по часовой стрелке.

Вариант N 56

Эллипс с центром $Point2D\left(2, -3\right)$, горизонтальной полуосью $36 \sqrt{11} / 11$, эксцентриситетом $5 / 6$.

    Изобразить на графике эллипс, а также эллипс, повернутый на угол $\alpha = $$\pi / 6$ против часовой стрелки.

Вариант N 57

Эллипс с центром $Point2D\left(-4, -4\right)$, вертикальной полуосью $2$, эксцентриситетом $3 / 4$.

    Изобразить на графике эллипс, а также эллипс, повернутый на угол $\alpha = $$\pi / 4$ по часовой стрелке.

Вариант N 58

Эллипс с центром $Point2D\left(3, 2\right)$, горизонтальной полуосью $21 \sqrt{2} / 4$, эксцентриситетом $1 / 3$.

    Изобразить на графике эллипс, а также эллипс, повернутый на угол $\alpha = $$\pi / 6$ против часовой стрелки.

Вариант N 59

Эллипс с центром $Point2D\left(5, -4\right)$, вертикальной полуосью $6$, эксцентриситетом $5 / 8$.

    Изобразить на графике эллипс, а также эллипс, повернутый на угол $\alpha = $$\pi / 6$ по часовой стрелке.

Вариант N 60

Эллипс с центром $Point2D\left(4, 0\right)$, горизонтальной полуосью $88 \sqrt{85} / 85$, эксцентриситетом $6 / 11$.

    Изобразить на графике эллипс, а также эллипс, повернутый на угол $\alpha = $$\pi / 3$ против часовой стрелки.

Вариант N 61

Эллипс с центром $Point2D\left(4, -4\right)$, вертикальной полуосью $8$, эксцентриситетом $3 / 5$.

    Изобразить на графике эллипс, а также эллипс, повернутый на угол $\alpha = $$\pi / 3$ по часовой стрелке.

Вариант N 62

Эллипс с центром $Point2D\left(3, 2\right)$, горизонтальной полуосью $45 \sqrt{14} / 28$, эксцентриситетом $5 / 9$.

    Изобразить на графике эллипс, а также эллипс, повернутый на угол $\alpha = $$\pi / 3$ против часовой стрелки.

Вариант N 63

Эллипс с центром $Point2D\left(1, -2\right)$, вертикальной полуосью $9$, эксцентриситетом $2 / 3$.

    Изобразить на графике эллипс, а также эллипс, повернутый на угол $\alpha = $$\pi / 4$ по часовой стрелке.

Вариант N 64

Эллипс с центром $Point2D\left(4, -1\right)$, горизонтальной полуосью $11 \sqrt{6} / 8$, эксцентриситетом $5 / 11$.

    Изобразить на графике эллипс, а также эллипс, повернутый на угол $\alpha = $$\pi / 4$ против часовой стрелки.

Вариант N 65

Эллипс с центром $Point2D\left(2, -1\right)$, вертикальной полуосью $7$, эксцентриситетом $5 / 8$.

    Изобразить на графике эллипс, а также эллипс, повернутый на угол $\alpha = $$\pi / 6$ по часовой стрелке.

Вариант N 66

Эллипс с центром $Point2D\left(4, -4\right)$, горизонтальной полуосью $55 \sqrt{2} / 12$, эксцентриситетом $7 / 11$.

    Изобразить на графике эллипс, а также эллипс, повернутый на угол $\alpha = $$\pi / 4$ против часовой стрелки.

Вариант N 67

Эллипс с центром $Point2D\left(-2, -4\right)$, вертикальной полуосью $8$, эксцентриситетом $9 / 10$.

    Изобразить на графике эллипс, а также эллипс, повернутый на угол $\alpha = $$\pi / 4$ по часовой стрелке.

Вариант N 68

Эллипс с центром $Point2D\left(3, -1\right)$, горизонтальной полуосью $32 \sqrt{7} / 7$, эксцентриситетом $3 / 4$.

    Изобразить на графике эллипс, а также эллипс, повернутый на угол $\alpha = $$\pi / 3$ против часовой стрелки.

Вариант N 69

Эллипс с центром $Point2D\left(0, -2\right)$, вертикальной полуосью $3$, эксцентриситетом $1 / 2$.

    Изобразить на графике эллипс, а также эллипс, повернутый на угол $\alpha = $$\pi / 3$ по часовой стрелке.

Вариант N 70

Эллипс с центром $Point2D\left(-2, 5\right)$, горизонтальной полуосью $8 \sqrt{3} / 3$, эксцентриситетом $1 / 2$.

    Изобразить на графике эллипс, а также эллипс, повернутый на угол $\alpha = $$\pi / 3$ против часовой стрелки.

Вариант N 71

Эллипс с центром $Point2D\left(-4, -2\right)$, вертикальной полуосью $7$, эксцентриситетом $8 / 17$.

    Изобразить на графике эллипс, а также эллипс, повернутый на угол $\alpha = $$\pi / 4$ по часовой стрелке.

Вариант N 72

Эллипс с центром $Point2D\left(-2, 1\right)$, горизонтальной полуосью $27 \sqrt{5} / 5$, эксцентриситетом $2 / 3$.

    Изобразить на графике эллипс, а также эллипс, повернутый на угол $\alpha = $$\pi / 3$ против часовой стрелки.

Вариант N 73

Эллипс с центром $Point2D\left(5, -3\right)$, вертикальной полуосью $6$, эксцентриситетом $7 / 9$.

    Изобразить на графике эллипс, а также эллипс, повернутый на угол $\alpha = $$\pi / 4$ по часовой стрелке.

Вариант N 74

Эллипс с центром $Point2D\left(-3, 4\right)$, горизонтальной полуосью $7 \sqrt{10} / 4$, эксцентриситетом $3 / 7$.

    Изобразить на графике эллипс, а также эллипс, повернутый на угол $\alpha = $$\pi / 4$ против часовой стрелки.

Вариант N 75

Эллипс с центром $Point2D\left(1, 4\right)$, вертикальной полуосью $7$, эксцентриситетом $7 / 13$.

    Изобразить на графике эллипс, а также эллипс, повернутый на угол $\alpha = $$\pi / 4$ по часовой стрелке.

Вариант N 76

Эллипс с центром $Point2D\left(3, 5\right)$, горизонтальной полуосью $51 / 5$, эксцентриситетом $8 / 17$.

    Изобразить на графике эллипс, а также эллипс, повернутый на угол $\alpha = $$\pi / 3$ против часовой стрелки.

Вариант N 77

Эллипс с центром $Point2D\left(3, 4\right)$, вертикальной полуосью $9$, эксцентриситетом $5 / 11$.

    Изобразить на графике эллипс, а также эллипс, повернутый на угол $\alpha = $$\pi / 4$ по часовой стрелке.

Вариант N 78

Эллипс с центром $Point2D\left(5, 1\right)$, горизонтальной полуосью $33 \sqrt{10} / 10$, эксцентриситетом $9 / 11$.

    Изобразить на графике эллипс, а также эллипс, повернутый на угол $\alpha = $$\pi / 6$ против часовой стрелки.

Вариант N 79

Эллипс с центром $Point2D\left(-3, -4\right)$, вертикальной полуосью $8$, эксцентриситетом $5 / 11$.

    Изобразить на графике эллипс, а также эллипс, повернутый на угол $\alpha = $$\pi / 4$ по часовой стрелке.

Вариант N 80

Эллипс с центром $Point2D\left(4, -1\right)$, горизонтальной полуосью $8 \sqrt{15} / 15$, эксцентриситетом $1 / 4$.

    Изобразить на графике эллипс, а также эллипс, повернутый на угол $\alpha = $$\pi / 6$ против часовой стрелки.

Вариант N 81

Эллипс с центром $Point2D\left(5, -3\right)$, вертикальной полуосью $3$, эксцентриситетом $7 / 15$.

    Изобразить на графике эллипс, а также эллипс, повернутый на угол $\alpha = $$\pi / 4$ по часовой стрелке.

Вариант N 82

Эллипс с центром $Point2D\left(-3, -4\right)$, горизонтальной полуосью $49 \sqrt{33} / 33$, эксцентриситетом $4 / 7$.

    Изобразить на графике эллипс, а также эллипс, повернутый на угол $\alpha = $$\pi / 4$ против часовой стрелки.

Вариант N 83

Эллипс с центром $Point2D\left(2, 3\right)$, вертикальной полуосью $9$, эксцентриситетом $2 / 5$.

    Изобразить на графике эллипс, а также эллипс, повернутый на угол $\alpha = $$\pi / 3$ по часовой стрелке.

Вариант N 84

Эллипс с центром $Point2D\left(4, -2\right)$, горизонтальной полуосью $20 \sqrt{7} / 7$, эксцентриситетом $3 / 4$.

    Изобразить на графике эллипс, а также эллипс, повернутый на угол $\alpha = $$\pi / 3$ против часовой стрелки.

Вариант N 85

Эллипс с центром $Point2D\left(-4, 2\right)$, вертикальной полуосью $5$, эксцентриситетом $4 / 5$.

    Изобразить на графике эллипс, а также эллипс, повернутый на угол $\alpha = $$\pi / 3$ по часовой стрелке.

Вариант N 86

Эллипс с центром $Point2D\left(1, 3\right)$, горизонтальной полуосью $25 \sqrt{6} / 12$, эксцентриситетом $1 / 5$.

    Изобразить на графике эллипс, а также эллипс, повернутый на угол $\alpha = $$\pi / 3$ против часовой стрелки.

Вариант N 87

Эллипс с центром $Point2D\left(-4, -1\right)$, вертикальной полуосью $2$, эксцентриситетом $7 / 11$.

    Изобразить на графике эллипс, а также эллипс, повернутый на угол $\alpha = $$\pi / 4$ по часовой стрелке.

Вариант N 88

Эллипс с центром $Point2D\left(-4, 3\right)$, горизонтальной полуосью $4 \sqrt{3}$, эксцентриситетом $1 / 2$.

    Изобразить на графике эллипс, а также эллипс, повернутый на угол $\alpha = $$\pi / 4$ против часовой стрелки.

Вариант N 89

Эллипс с центром $Point2D\left(1, 2\right)$, вертикальной полуосью $8$, эксцентриситетом $1 / 3$.

    Изобразить на графике эллипс, а также эллипс, повернутый на угол $\alpha = $$\pi / 4$ по часовой стрелке.

Вариант N 90

Эллипс с центром $Point2D\left(0, 4\right)$, горизонтальной полуосью $10 \sqrt{21} / 7$, эксцентриситетом $2 / 5$.

    Изобразить на графике эллипс, а также эллипс, повернутый на угол $\alpha = $$\pi / 6$ против часовой стрелки.

Вариант N 91

Эллипс с центром $Point2D\left(-3, -4\right)$, вертикальной полуосью $4$, эксцентриситетом $1 / 5$.

    Изобразить на графике эллипс, а также эллипс, повернутый на угол $\alpha = $$\pi / 3$ по часовой стрелке.

Вариант N 92

Эллипс с центром $Point2D\left(2, -3\right)$, горизонтальной полуосью $7 \sqrt{33} / 11$, эксцентриситетом $4 / 7$.

    Изобразить на графике эллипс, а также эллипс, повернутый на угол $\alpha = $$\pi / 6$ против часовой стрелки.

Вариант N 93

Эллипс с центром $Point2D\left(4, -3\right)$, вертикальной полуосью $4$, эксцентриситетом $1 / 3$.

    Изобразить на графике эллипс, а также эллипс, повернутый на угол $\alpha = $$\pi / 3$ по часовой стрелке.

Вариант N 94

Эллипс с центром $Point2D\left(1, -2\right)$, горизонтальной полуосью $39 \sqrt{22} / 44$, эксцентриситетом $9 / 13$.

    Изобразить на графике эллипс, а также эллипс, повернутый на угол $\alpha = $$\pi / 4$ против часовой стрелки.

Вариант N 95

Эллипс с центром $Point2D\left(-4, -2\right)$, вертикальной полуосью $4$, эксцентриситетом $4 / 9$.

    Изобразить на графике эллипс, а также эллипс, повернутый на угол $\alpha = $$\pi / 4$ по часовой стрелке.

Вариант N 96

Эллипс с центром $Point2D\left(2, 3\right)$, горизонтальной полуосью $8 \sqrt{7} / 7$, эксцентриситетом $3 / 4$.

    Изобразить на графике эллипс, а также эллипс, повернутый на угол $\alpha = $$\pi / 3$ против часовой стрелки.

Вариант N 97

Эллипс с центром $Point2D\left(1, -1\right)$, вертикальной полуосью $2$, эксцентриситетом $4 / 7$.

    Изобразить на графике эллипс, а также эллипс, повернутый на угол $\alpha = $$\pi / 6$ по часовой стрелке.

Вариант N 98

Эллипс с центром $Point2D\left(0, 0\right)$, горизонтальной полуосью $22 \sqrt{85} / 85$, эксцентриситетом $6 / 11$.

    Изобразить на графике эллипс, а также эллипс, повернутый на угол $\alpha = $$\pi / 3$ против часовой стрелки.

Вариант N 99

Эллипс с центром $Point2D\left(1, 2\right)$, вертикальной полуосью $3$, эксцентриситетом $3 / 5$.

    Изобразить на графике эллипс, а также эллипс, повернутый на угол $\alpha = $$\pi / 3$ по часовой стрелке.

Вариант N 100

Эллипс с центром $Point2D\left(-1, 1\right)$, горизонтальной полуосью $28 \sqrt{15} / 15$, эксцентриситетом $1 / 4$.

    Изобразить на графике эллипс, а также эллипс, повернутый на угол $\alpha = $$\pi / 3$ против часовой стрелки.

Вариант N 101

Эллипс с центром $Point2D\left(5, -2\right)$, вертикальной полуосью $9$, эксцентриситетом $4 / 7$.

    Изобразить на графике эллипс, а также эллипс, повернутый на угол $\alpha = $$\pi / 3$ по часовой стрелке.

Вариант N 102

Эллипс с центром $Point2D\left(0, 2\right)$, горизонтальной полуосью $36 \sqrt{7} / 7$, эксцентриситетом $3 / 4$.

    Изобразить на графике эллипс, а также эллипс, повернутый на угол $\alpha = $$\pi / 6$ против часовой стрелки.

Вариант N 103

Эллипс с центром $Point2D\left(-2, -4\right)$, вертикальной полуосью $2$, эксцентриситетом $3 / 7$.

    Изобразить на графике эллипс, а также эллипс, повернутый на угол $\alpha = $$\pi / 3$ по часовой стрелке.

Вариант N 104

Эллипс с центром $Point2D\left(-4, -2\right)$, горизонтальной полуосью $20 \sqrt{21} / 21$, эксцентриситетом $2 / 5$.

    Изобразить на графике эллипс, а также эллипс, повернутый на угол $\alpha = $$\pi / 6$ против часовой стрелки.

Вариант N 105

Эллипс с центром $Point2D\left(3, 4\right)$, вертикальной полуосью $8$, эксцентриситетом $2 / 5$.

    Изобразить на графике эллипс, а также эллипс, повернутый на угол $\alpha = $$\pi / 3$ по часовой стрелке.

Вариант N 106

Эллипс с центром $Point2D\left(-3, 2\right)$, горизонтальной полуосью $8 \sqrt{7} / 7$, эксцентриситетом $3 / 4$.

    Изобразить на графике эллипс, а также эллипс, повернутый на угол $\alpha = $$\pi / 3$ против часовой стрелки.

Вариант N 107

Эллипс с центром $Point2D\left(5, 2\right)$, вертикальной полуосью $7$, эксцентриситетом $1 / 2$.

    Изобразить на графике эллипс, а также эллипс, повернутый на угол $\alpha = $$\pi / 4$ по часовой стрелке.

Вариант N 108

Эллипс с центром $Point2D\left(1, -2\right)$, горизонтальной полуосью $81 \sqrt{17} / 17$, эксцентриситетом $8 / 9$.

    Изобразить на графике эллипс, а также эллипс, повернутый на угол $\alpha = $$\pi / 4$ против часовой стрелки.

Вариант N 109

Эллипс с центром $Point2D\left(1, 5\right)$, вертикальной полуосью $9$, эксцентриситетом $5 / 14$.

    Изобразить на графике эллипс, а также эллипс, повернутый на угол $\alpha = $$\pi / 3$ по часовой стрелке.

Вариант N 110

Эллипс с центром $Point2D\left(5, -3\right)$, горизонтальной полуосью $22 \sqrt{7} / 7$, эксцентриситетом $3 / 11$.

    Изобразить на графике эллипс, а также эллипс, повернутый на угол $\alpha = $$\pi / 3$ против часовой стрелки.

Вариант N 111

Эллипс с центром $Point2D\left(3, 4\right)$, вертикальной полуосью $8$, эксцентриситетом $4 / 7$.

    Изобразить на графике эллипс, а также эллипс, повернутый на угол $\alpha = $$\pi / 6$ по часовой стрелке.

Вариант N 112

Эллипс с центром $Point2D\left(4, 4\right)$, горизонтальной полуосью $6 \sqrt{5} / 5$, эксцентриситетом $2 / 3$.

    Изобразить на графике эллипс, а также эллипс, повернутый на угол $\alpha = $$\pi / 4$ против часовой стрелки.

Вариант N 113

Эллипс с центром $Point2D\left(4, 2\right)$, вертикальной полуосью $7$, эксцентриситетом $2 / 3$.

    Изобразить на графике эллипс, а также эллипс, повернутый на угол $\alpha = $$\pi / 3$ по часовой стрелке.

Вариант N 114

Эллипс с центром $Point2D\left(5, -3\right)$, горизонтальной полуосью $45 \sqrt{77} / 77$, эксцентриситетом $2 / 9$.

    Изобразить на графике эллипс, а также эллипс, повернутый на угол $\alpha = $$\pi / 4$ против часовой стрелки.

Вариант N 115

Эллипс с центром $Point2D\left(5, -1\right)$, вертикальной полуосью $5$, эксцентриситетом $9 / 14$.

    Изобразить на графике эллипс, а также эллипс, повернутый на угол $\alpha = $$\pi / 3$ по часовой стрелке.

Вариант N 116

Эллипс с центром $Point2D\left(2, -4\right)$, горизонтальной полуосью $28 \sqrt{15} / 15$, эксцентриситетом $1 / 4$.

    Изобразить на графике эллипс, а также эллипс, повернутый на угол $\alpha = $$\pi / 3$ против часовой стрелки.

Вариант N 117

Эллипс с центром $Point2D\left(-3, -1\right)$, вертикальной полуосью $8$, эксцентриситетом $4 / 13$.

    Изобразить на графике эллипс, а также эллипс, повернутый на угол $\alpha = $$\pi / 4$ по часовой стрелке.

Вариант N 118

Эллипс с центром $Point2D\left(4, 3\right)$, горизонтальной полуосью $11 \sqrt{2} / 3$, эксцентриситетом $7 / 11$.

    Изобразить на графике эллипс, а также эллипс, повернутый на угол $\alpha = $$\pi / 3$ против часовой стрелки.

Вариант N 119

Эллипс с центром $Point2D\left(2, 1\right)$, вертикальной полуосью $8$, эксцентриситетом $3 / 4$.

    Изобразить на графике эллипс, а также эллипс, повернутый на угол $\alpha = $$\pi / 4$ по часовой стрелке.

Вариант N 120

Эллипс с центром $Point2D\left(5, -3\right)$, горизонтальной полуосью $3 \sqrt{2} / 2$, эксцентриситетом $1 / 3$.

    Изобразить на графике эллипс, а также эллипс, повернутый на угол $\alpha = $$\pi / 6$ против часовой стрелки.

Вариант N 121

Эллипс с центром $Point2D\left(2, 3\right)$, вертикальной полуосью $4$, эксцентриситетом $5 / 13$.

    Изобразить на графике эллипс, а также эллипс, повернутый на угол $\alpha = $$\pi / 6$ по часовой стрелке.

Вариант N 122

Эллипс с центром $Point2D\left(-2, 0\right)$, горизонтальной полуосью $81 \sqrt{17} / 17$, эксцентриситетом $8 / 9$.

    Изобразить на графике эллипс, а также эллипс, повернутый на угол $\alpha = $$\pi / 3$ против часовой стрелки.

Вариант N 123

Эллипс с центром $Point2D\left(3, 4\right)$, вертикальной полуосью $2$, эксцентриситетом $2 / 9$.

    Изобразить на графике эллипс, а также эллипс, повернутый на угол $\alpha = $$\pi / 6$ по часовой стрелке.

Вариант N 124

Эллипс с центром $Point2D\left(1, 2\right)$, горизонтальной полуосью $15 / 2$, эксцентриситетом $3 / 5$.

    Изобразить на графике эллипс, а также эллипс, повернутый на угол $\alpha = $$\pi / 4$ против часовой стрелки.

Вариант N 125

Эллипс с центром $Point2D\left(-4, -3\right)$, вертикальной полуосью $2$, эксцентриситетом $6 / 7$.

    Изобразить на графике эллипс, а также эллипс, повернутый на угол $\alpha = $$\pi / 6$ по часовой стрелке.

Вариант N 126

Эллипс с центром $Point2D\left(2, 4\right)$, горизонтальной полуосью $8 \sqrt{3} / 3$, эксцентриситетом $1 / 2$.

    Изобразить на графике эллипс, а также эллипс, повернутый на угол $\alpha = $$\pi / 6$ против часовой стрелки.

Вариант N 127

Эллипс с центром $Point2D\left(-3, -1\right)$, вертикальной полуосью $9$, эксцентриситетом $4 / 13$.

    Изобразить на графике эллипс, а также эллипс, повернутый на угол $\alpha = $$\pi / 4$ по часовой стрелке.

Вариант N 128

Эллипс с центром $Point2D\left(0, 2\right)$, горизонтальной полуосью $24 \sqrt{7} / 7$, эксцентриситетом $3 / 4$.

    Изобразить на графике эллипс, а также эллипс, повернутый на угол $\alpha = $$\pi / 3$ против часовой стрелки.

Вариант N 129

Эллипс с центром $Point2D\left(3, -1\right)$, вертикальной полуосью $3$, эксцентриситетом $1 / 3$.

    Изобразить на графике эллипс, а также эллипс, повернутый на угол $\alpha = $$\pi / 4$ по часовой стрелке.

Вариант N 130

Эллипс с центром $Point2D\left(2, 4\right)$, горизонтальной полуосью $136 / 15$, эксцентриситетом $8 / 17$.

    Изобразить на графике эллипс, а также эллипс, повернутый на угол $\alpha = $$\pi / 3$ против часовой стрелки.

Вариант N 131

Эллипс с центром $Point2D\left(-3, 0\right)$, вертикальной полуосью $9$, эксцентриситетом $3 / 11$.

    Изобразить на графике эллипс, а также эллипс, повернутый на угол $\alpha = $$\pi / 4$ по часовой стрелке.

Вариант N 132

Эллипс с центром $Point2D\left(0, 5\right)$, горизонтальной полуосью $24 \sqrt{5} / 5$, эксцентриситетом $2 / 3$.

    Изобразить на графике эллипс, а также эллипс, повернутый на угол $\alpha = $$\pi / 3$ против часовой стрелки.

Вариант N 133

Эллипс с центром $Point2D\left(3, 1\right)$, вертикальной полуосью $8$, эксцентриситетом $1 / 5$.

    Изобразить на графике эллипс, а также эллипс, повернутый на угол $\alpha = $$\pi / 3$ по часовой стрелке.

Вариант N 134

Эллипс с центром $Point2D\left(-3, 1\right)$, горизонтальной полуосью $11 \sqrt{2} / 3$, эксцентриситетом $7 / 11$.

    Изобразить на графике эллипс, а также эллипс, повернутый на угол $\alpha = $$\pi / 4$ против часовой стрелки.

Вариант N 135

Эллипс с центром $Point2D\left(-2, -1\right)$, вертикальной полуосью $4$, эксцентриситетом $5 / 8$.

    Изобразить на графике эллипс, а также эллипс, повернутый на угол $\alpha = $$\pi / 3$ по часовой стрелке.

Вариант N 136

Эллипс с центром $Point2D\left(4, 1\right)$, горизонтальной полуосью $25 / 4$, эксцентриситетом $3 / 5$.

    Изобразить на графике эллипс, а также эллипс, повернутый на угол $\alpha = $$\pi / 6$ против часовой стрелки.

Вариант N 137

Эллипс с центром $Point2D\left(2, -4\right)$, вертикальной полуосью $7$, эксцентриситетом $7 / 8$.

    Изобразить на графике эллипс, а также эллипс, повернутый на угол $\alpha = $$\pi / 6$ по часовой стрелке.

Вариант N 138

Эллипс с центром $Point2D\left(-2, 3\right)$, горизонтальной полуосью $9 \sqrt{77} / 11$, эксцентриситетом $2 / 9$.

    Изобразить на графике эллипс, а также эллипс, повернутый на угол $\alpha = $$\pi / 4$ против часовой стрелки.

Вариант N 139

Эллипс с центром $Point2D\left(3, 2\right)$, вертикальной полуосью $6$, эксцентриситетом $2 / 3$.

    Изобразить на графике эллипс, а также эллипс, повернутый на угол $\alpha = $$\pi / 3$ по часовой стрелке.

Вариант N 140

Эллипс с центром $Point2D\left(2, -2\right)$, горизонтальной полуосью $21 \sqrt{10} / 20$, эксцентриситетом $3 / 7$.

    Изобразить на графике эллипс, а также эллипс, повернутый на угол $\alpha = $$\pi / 6$ против часовой стрелки.

Вариант N 141

Эллипс с центром $Point2D\left(1, -4\right)$, вертикальной полуосью $2$, эксцентриситетом $2 / 5$.

    Изобразить на графике эллипс, а также эллипс, повернутый на угол $\alpha = $$\pi / 4$ по часовой стрелке.

Вариант N 142

Эллипс с центром $Point2D\left(5, -3\right)$, горизонтальной полуосью $14 \sqrt{6} / 3$, эксцентриситетом $5 / 7$.

    Изобразить на графике эллипс, а также эллипс, повернутый на угол $\alpha = $$\pi / 3$ против часовой стрелки.

Вариант N 143

Эллипс с центром $Point2D\left(-2, -1\right)$, вертикальной полуосью $2$, эксцентриситетом $1 / 2$.

    Изобразить на графике эллипс, а также эллипс, повернутый на угол $\alpha = $$\pi / 6$ по часовой стрелке.

Вариант N 144

Эллипс с центром $Point2D\left(-1, -4\right)$, горизонтальной полуосью $55 \sqrt{2} / 12$, эксцентриситетом $7 / 11$.

    Изобразить на графике эллипс, а также эллипс, повернутый на угол $\alpha = $$\pi / 3$ против часовой стрелки.

Вариант N 145

Эллипс с центром $Point2D\left(-4, -2\right)$, вертикальной полуосью $7$, эксцентриситетом $3 / 5$.

    Изобразить на графике эллипс, а также эллипс, повернутый на угол $\alpha = $$\pi / 6$ по часовой стрелке.

Вариант N 146

Эллипс с центром $Point2D\left(-4, 3\right)$, горизонтальной полуосью $22 \sqrt{85} / 85$, эксцентриситетом $6 / 11$.

    Изобразить на графике эллипс, а также эллипс, повернутый на угол $\alpha = $$\pi / 4$ против часовой стрелки.

Вариант N 147

Эллипс с центром $Point2D\left(-3, 1\right)$, вертикальной полуосью $5$, эксцентриситетом $1 / 2$.

    Изобразить на графике эллипс, а также эллипс, повернутый на угол $\alpha = $$\pi / 3$ по часовой стрелке.

Вариант N 148

Эллипс с центром $Point2D\left(0, -1\right)$, горизонтальной полуосью $16 \sqrt{7} / 5$, эксцентриситетом $9 / 16$.

    Изобразить на графике эллипс, а также эллипс, повернутый на угол $\alpha = $$\pi / 6$ против часовой стрелки.

Вариант N 149

Эллипс с центром $Point2D\left(-3, 3\right)$, вертикальной полуосью $9$, эксцентриситетом $9 / 13$.

    Изобразить на графике эллипс, а также эллипс, повернутый на угол $\alpha = $$\pi / 4$ по часовой стрелке.

Вариант N 150

Эллипс с центром $Point2D\left(0, -1\right)$, горизонтальной полуосью $16 \sqrt{15} / 5$, эксцентриситетом $7 / 8$.

    Изобразить на графике эллипс, а также эллипс, повернутый на угол $\alpha = $$\pi / 6$ против часовой стрелки.

\end{document}